
% Default to the notebook output style

    


% Inherit from the specified cell style.




    
\documentclass[11pt]{article}

    
    
    \usepackage[T1]{fontenc}
    % Nicer default font (+ math font) than Computer Modern for most use cases
    \usepackage{mathpazo}

    % Basic figure setup, for now with no caption control since it's done
    % automatically by Pandoc (which extracts ![](path) syntax from Markdown).
    \usepackage{graphicx}
    % We will generate all images so they have a width \maxwidth. This means
    % that they will get their normal width if they fit onto the page, but
    % are scaled down if they would overflow the margins.
    \makeatletter
    \def\maxwidth{\ifdim\Gin@nat@width>\linewidth\linewidth
    \else\Gin@nat@width\fi}
    \makeatother
    \let\Oldincludegraphics\includegraphics
    % Set max figure width to be 80% of text width, for now hardcoded.
    \renewcommand{\includegraphics}[1]{\Oldincludegraphics[width=.8\maxwidth]{#1}}
    % Ensure that by default, figures have no caption (until we provide a
    % proper Figure object with a Caption API and a way to capture that
    % in the conversion process - todo).
    \usepackage{caption}
    \DeclareCaptionLabelFormat{nolabel}{}
    \captionsetup{labelformat=nolabel}

    \usepackage{adjustbox} % Used to constrain images to a maximum size 
    \usepackage{xcolor} % Allow colors to be defined
    \usepackage{enumerate} % Needed for markdown enumerations to work
    \usepackage{geometry} % Used to adjust the document margins
    \usepackage{amsmath} % Equations
    \usepackage{amssymb} % Equations
    \usepackage{textcomp} % defines textquotesingle
    % Hack from http://tex.stackexchange.com/a/47451/13684:
    \AtBeginDocument{%
        \def\PYZsq{\textquotesingle}% Upright quotes in Pygmentized code
    }
    \usepackage{upquote} % Upright quotes for verbatim code
    \usepackage{eurosym} % defines \euro
    \usepackage[mathletters]{ucs} % Extended unicode (utf-8) support
    \usepackage[utf8x]{inputenc} % Allow utf-8 characters in the tex document
    \usepackage{fancyvrb} % verbatim replacement that allows latex
    \usepackage{grffile} % extends the file name processing of package graphics 
                         % to support a larger range 
    % The hyperref package gives us a pdf with properly built
    % internal navigation ('pdf bookmarks' for the table of contents,
    % internal cross-reference links, web links for URLs, etc.)
    \usepackage{hyperref}
    \usepackage{longtable} % longtable support required by pandoc >1.10
    \usepackage{booktabs}  % table support for pandoc > 1.12.2
    \usepackage[inline]{enumitem} % IRkernel/repr support (it uses the enumerate* environment)
    \usepackage[normalem]{ulem} % ulem is needed to support strikethroughs (\sout)
                                % normalem makes italics be italics, not underlines
    

    
    
    % Colors for the hyperref package
    \definecolor{urlcolor}{rgb}{0,.145,.698}
    \definecolor{linkcolor}{rgb}{.71,0.21,0.01}
    \definecolor{citecolor}{rgb}{.12,.54,.11}

    % ANSI colors
    \definecolor{ansi-black}{HTML}{3E424D}
    \definecolor{ansi-black-intense}{HTML}{282C36}
    \definecolor{ansi-red}{HTML}{E75C58}
    \definecolor{ansi-red-intense}{HTML}{B22B31}
    \definecolor{ansi-green}{HTML}{00A250}
    \definecolor{ansi-green-intense}{HTML}{007427}
    \definecolor{ansi-yellow}{HTML}{DDB62B}
    \definecolor{ansi-yellow-intense}{HTML}{B27D12}
    \definecolor{ansi-blue}{HTML}{208FFB}
    \definecolor{ansi-blue-intense}{HTML}{0065CA}
    \definecolor{ansi-magenta}{HTML}{D160C4}
    \definecolor{ansi-magenta-intense}{HTML}{A03196}
    \definecolor{ansi-cyan}{HTML}{60C6C8}
    \definecolor{ansi-cyan-intense}{HTML}{258F8F}
    \definecolor{ansi-white}{HTML}{C5C1B4}
    \definecolor{ansi-white-intense}{HTML}{A1A6B2}

    % commands and environments needed by pandoc snippets
    % extracted from the output of `pandoc -s`
    \providecommand{\tightlist}{%
      \setlength{\itemsep}{0pt}\setlength{\parskip}{0pt}}
    \DefineVerbatimEnvironment{Highlighting}{Verbatim}{commandchars=\\\{\}}
    % Add ',fontsize=\small' for more characters per line
    \newenvironment{Shaded}{}{}
    \newcommand{\KeywordTok}[1]{\textcolor[rgb]{0.00,0.44,0.13}{\textbf{{#1}}}}
    \newcommand{\DataTypeTok}[1]{\textcolor[rgb]{0.56,0.13,0.00}{{#1}}}
    \newcommand{\DecValTok}[1]{\textcolor[rgb]{0.25,0.63,0.44}{{#1}}}
    \newcommand{\BaseNTok}[1]{\textcolor[rgb]{0.25,0.63,0.44}{{#1}}}
    \newcommand{\FloatTok}[1]{\textcolor[rgb]{0.25,0.63,0.44}{{#1}}}
    \newcommand{\CharTok}[1]{\textcolor[rgb]{0.25,0.44,0.63}{{#1}}}
    \newcommand{\StringTok}[1]{\textcolor[rgb]{0.25,0.44,0.63}{{#1}}}
    \newcommand{\CommentTok}[1]{\textcolor[rgb]{0.38,0.63,0.69}{\textit{{#1}}}}
    \newcommand{\OtherTok}[1]{\textcolor[rgb]{0.00,0.44,0.13}{{#1}}}
    \newcommand{\AlertTok}[1]{\textcolor[rgb]{1.00,0.00,0.00}{\textbf{{#1}}}}
    \newcommand{\FunctionTok}[1]{\textcolor[rgb]{0.02,0.16,0.49}{{#1}}}
    \newcommand{\RegionMarkerTok}[1]{{#1}}
    \newcommand{\ErrorTok}[1]{\textcolor[rgb]{1.00,0.00,0.00}{\textbf{{#1}}}}
    \newcommand{\NormalTok}[1]{{#1}}
    
    % Additional commands for more recent versions of Pandoc
    \newcommand{\ConstantTok}[1]{\textcolor[rgb]{0.53,0.00,0.00}{{#1}}}
    \newcommand{\SpecialCharTok}[1]{\textcolor[rgb]{0.25,0.44,0.63}{{#1}}}
    \newcommand{\VerbatimStringTok}[1]{\textcolor[rgb]{0.25,0.44,0.63}{{#1}}}
    \newcommand{\SpecialStringTok}[1]{\textcolor[rgb]{0.73,0.40,0.53}{{#1}}}
    \newcommand{\ImportTok}[1]{{#1}}
    \newcommand{\DocumentationTok}[1]{\textcolor[rgb]{0.73,0.13,0.13}{\textit{{#1}}}}
    \newcommand{\AnnotationTok}[1]{\textcolor[rgb]{0.38,0.63,0.69}{\textbf{\textit{{#1}}}}}
    \newcommand{\CommentVarTok}[1]{\textcolor[rgb]{0.38,0.63,0.69}{\textbf{\textit{{#1}}}}}
    \newcommand{\VariableTok}[1]{\textcolor[rgb]{0.10,0.09,0.49}{{#1}}}
    \newcommand{\ControlFlowTok}[1]{\textcolor[rgb]{0.00,0.44,0.13}{\textbf{{#1}}}}
    \newcommand{\OperatorTok}[1]{\textcolor[rgb]{0.40,0.40,0.40}{{#1}}}
    \newcommand{\BuiltInTok}[1]{{#1}}
    \newcommand{\ExtensionTok}[1]{{#1}}
    \newcommand{\PreprocessorTok}[1]{\textcolor[rgb]{0.74,0.48,0.00}{{#1}}}
    \newcommand{\AttributeTok}[1]{\textcolor[rgb]{0.49,0.56,0.16}{{#1}}}
    \newcommand{\InformationTok}[1]{\textcolor[rgb]{0.38,0.63,0.69}{\textbf{\textit{{#1}}}}}
    \newcommand{\WarningTok}[1]{\textcolor[rgb]{0.38,0.63,0.69}{\textbf{\textit{{#1}}}}}
    
    
    % Define a nice break command that doesn't care if a line doesn't already
    % exist.
    \def\br{\hspace*{\fill} \\* }
    % Math Jax compatability definitions
    \def\gt{>}
    \def\lt{<}
    % Document parameters
    \title{hw1}
    
    
    

    % Pygments definitions
    
\makeatletter
\def\PY@reset{\let\PY@it=\relax \let\PY@bf=\relax%
    \let\PY@ul=\relax \let\PY@tc=\relax%
    \let\PY@bc=\relax \let\PY@ff=\relax}
\def\PY@tok#1{\csname PY@tok@#1\endcsname}
\def\PY@toks#1+{\ifx\relax#1\empty\else%
    \PY@tok{#1}\expandafter\PY@toks\fi}
\def\PY@do#1{\PY@bc{\PY@tc{\PY@ul{%
    \PY@it{\PY@bf{\PY@ff{#1}}}}}}}
\def\PY#1#2{\PY@reset\PY@toks#1+\relax+\PY@do{#2}}

\expandafter\def\csname PY@tok@w\endcsname{\def\PY@tc##1{\textcolor[rgb]{0.73,0.73,0.73}{##1}}}
\expandafter\def\csname PY@tok@c\endcsname{\let\PY@it=\textit\def\PY@tc##1{\textcolor[rgb]{0.25,0.50,0.50}{##1}}}
\expandafter\def\csname PY@tok@cp\endcsname{\def\PY@tc##1{\textcolor[rgb]{0.74,0.48,0.00}{##1}}}
\expandafter\def\csname PY@tok@k\endcsname{\let\PY@bf=\textbf\def\PY@tc##1{\textcolor[rgb]{0.00,0.50,0.00}{##1}}}
\expandafter\def\csname PY@tok@kp\endcsname{\def\PY@tc##1{\textcolor[rgb]{0.00,0.50,0.00}{##1}}}
\expandafter\def\csname PY@tok@kt\endcsname{\def\PY@tc##1{\textcolor[rgb]{0.69,0.00,0.25}{##1}}}
\expandafter\def\csname PY@tok@o\endcsname{\def\PY@tc##1{\textcolor[rgb]{0.40,0.40,0.40}{##1}}}
\expandafter\def\csname PY@tok@ow\endcsname{\let\PY@bf=\textbf\def\PY@tc##1{\textcolor[rgb]{0.67,0.13,1.00}{##1}}}
\expandafter\def\csname PY@tok@nb\endcsname{\def\PY@tc##1{\textcolor[rgb]{0.00,0.50,0.00}{##1}}}
\expandafter\def\csname PY@tok@nf\endcsname{\def\PY@tc##1{\textcolor[rgb]{0.00,0.00,1.00}{##1}}}
\expandafter\def\csname PY@tok@nc\endcsname{\let\PY@bf=\textbf\def\PY@tc##1{\textcolor[rgb]{0.00,0.00,1.00}{##1}}}
\expandafter\def\csname PY@tok@nn\endcsname{\let\PY@bf=\textbf\def\PY@tc##1{\textcolor[rgb]{0.00,0.00,1.00}{##1}}}
\expandafter\def\csname PY@tok@ne\endcsname{\let\PY@bf=\textbf\def\PY@tc##1{\textcolor[rgb]{0.82,0.25,0.23}{##1}}}
\expandafter\def\csname PY@tok@nv\endcsname{\def\PY@tc##1{\textcolor[rgb]{0.10,0.09,0.49}{##1}}}
\expandafter\def\csname PY@tok@no\endcsname{\def\PY@tc##1{\textcolor[rgb]{0.53,0.00,0.00}{##1}}}
\expandafter\def\csname PY@tok@nl\endcsname{\def\PY@tc##1{\textcolor[rgb]{0.63,0.63,0.00}{##1}}}
\expandafter\def\csname PY@tok@ni\endcsname{\let\PY@bf=\textbf\def\PY@tc##1{\textcolor[rgb]{0.60,0.60,0.60}{##1}}}
\expandafter\def\csname PY@tok@na\endcsname{\def\PY@tc##1{\textcolor[rgb]{0.49,0.56,0.16}{##1}}}
\expandafter\def\csname PY@tok@nt\endcsname{\let\PY@bf=\textbf\def\PY@tc##1{\textcolor[rgb]{0.00,0.50,0.00}{##1}}}
\expandafter\def\csname PY@tok@nd\endcsname{\def\PY@tc##1{\textcolor[rgb]{0.67,0.13,1.00}{##1}}}
\expandafter\def\csname PY@tok@s\endcsname{\def\PY@tc##1{\textcolor[rgb]{0.73,0.13,0.13}{##1}}}
\expandafter\def\csname PY@tok@sd\endcsname{\let\PY@it=\textit\def\PY@tc##1{\textcolor[rgb]{0.73,0.13,0.13}{##1}}}
\expandafter\def\csname PY@tok@si\endcsname{\let\PY@bf=\textbf\def\PY@tc##1{\textcolor[rgb]{0.73,0.40,0.53}{##1}}}
\expandafter\def\csname PY@tok@se\endcsname{\let\PY@bf=\textbf\def\PY@tc##1{\textcolor[rgb]{0.73,0.40,0.13}{##1}}}
\expandafter\def\csname PY@tok@sr\endcsname{\def\PY@tc##1{\textcolor[rgb]{0.73,0.40,0.53}{##1}}}
\expandafter\def\csname PY@tok@ss\endcsname{\def\PY@tc##1{\textcolor[rgb]{0.10,0.09,0.49}{##1}}}
\expandafter\def\csname PY@tok@sx\endcsname{\def\PY@tc##1{\textcolor[rgb]{0.00,0.50,0.00}{##1}}}
\expandafter\def\csname PY@tok@m\endcsname{\def\PY@tc##1{\textcolor[rgb]{0.40,0.40,0.40}{##1}}}
\expandafter\def\csname PY@tok@gh\endcsname{\let\PY@bf=\textbf\def\PY@tc##1{\textcolor[rgb]{0.00,0.00,0.50}{##1}}}
\expandafter\def\csname PY@tok@gu\endcsname{\let\PY@bf=\textbf\def\PY@tc##1{\textcolor[rgb]{0.50,0.00,0.50}{##1}}}
\expandafter\def\csname PY@tok@gd\endcsname{\def\PY@tc##1{\textcolor[rgb]{0.63,0.00,0.00}{##1}}}
\expandafter\def\csname PY@tok@gi\endcsname{\def\PY@tc##1{\textcolor[rgb]{0.00,0.63,0.00}{##1}}}
\expandafter\def\csname PY@tok@gr\endcsname{\def\PY@tc##1{\textcolor[rgb]{1.00,0.00,0.00}{##1}}}
\expandafter\def\csname PY@tok@ge\endcsname{\let\PY@it=\textit}
\expandafter\def\csname PY@tok@gs\endcsname{\let\PY@bf=\textbf}
\expandafter\def\csname PY@tok@gp\endcsname{\let\PY@bf=\textbf\def\PY@tc##1{\textcolor[rgb]{0.00,0.00,0.50}{##1}}}
\expandafter\def\csname PY@tok@go\endcsname{\def\PY@tc##1{\textcolor[rgb]{0.53,0.53,0.53}{##1}}}
\expandafter\def\csname PY@tok@gt\endcsname{\def\PY@tc##1{\textcolor[rgb]{0.00,0.27,0.87}{##1}}}
\expandafter\def\csname PY@tok@err\endcsname{\def\PY@bc##1{\setlength{\fboxsep}{0pt}\fcolorbox[rgb]{1.00,0.00,0.00}{1,1,1}{\strut ##1}}}
\expandafter\def\csname PY@tok@kc\endcsname{\let\PY@bf=\textbf\def\PY@tc##1{\textcolor[rgb]{0.00,0.50,0.00}{##1}}}
\expandafter\def\csname PY@tok@kd\endcsname{\let\PY@bf=\textbf\def\PY@tc##1{\textcolor[rgb]{0.00,0.50,0.00}{##1}}}
\expandafter\def\csname PY@tok@kn\endcsname{\let\PY@bf=\textbf\def\PY@tc##1{\textcolor[rgb]{0.00,0.50,0.00}{##1}}}
\expandafter\def\csname PY@tok@kr\endcsname{\let\PY@bf=\textbf\def\PY@tc##1{\textcolor[rgb]{0.00,0.50,0.00}{##1}}}
\expandafter\def\csname PY@tok@bp\endcsname{\def\PY@tc##1{\textcolor[rgb]{0.00,0.50,0.00}{##1}}}
\expandafter\def\csname PY@tok@fm\endcsname{\def\PY@tc##1{\textcolor[rgb]{0.00,0.00,1.00}{##1}}}
\expandafter\def\csname PY@tok@vc\endcsname{\def\PY@tc##1{\textcolor[rgb]{0.10,0.09,0.49}{##1}}}
\expandafter\def\csname PY@tok@vg\endcsname{\def\PY@tc##1{\textcolor[rgb]{0.10,0.09,0.49}{##1}}}
\expandafter\def\csname PY@tok@vi\endcsname{\def\PY@tc##1{\textcolor[rgb]{0.10,0.09,0.49}{##1}}}
\expandafter\def\csname PY@tok@vm\endcsname{\def\PY@tc##1{\textcolor[rgb]{0.10,0.09,0.49}{##1}}}
\expandafter\def\csname PY@tok@sa\endcsname{\def\PY@tc##1{\textcolor[rgb]{0.73,0.13,0.13}{##1}}}
\expandafter\def\csname PY@tok@sb\endcsname{\def\PY@tc##1{\textcolor[rgb]{0.73,0.13,0.13}{##1}}}
\expandafter\def\csname PY@tok@sc\endcsname{\def\PY@tc##1{\textcolor[rgb]{0.73,0.13,0.13}{##1}}}
\expandafter\def\csname PY@tok@dl\endcsname{\def\PY@tc##1{\textcolor[rgb]{0.73,0.13,0.13}{##1}}}
\expandafter\def\csname PY@tok@s2\endcsname{\def\PY@tc##1{\textcolor[rgb]{0.73,0.13,0.13}{##1}}}
\expandafter\def\csname PY@tok@sh\endcsname{\def\PY@tc##1{\textcolor[rgb]{0.73,0.13,0.13}{##1}}}
\expandafter\def\csname PY@tok@s1\endcsname{\def\PY@tc##1{\textcolor[rgb]{0.73,0.13,0.13}{##1}}}
\expandafter\def\csname PY@tok@mb\endcsname{\def\PY@tc##1{\textcolor[rgb]{0.40,0.40,0.40}{##1}}}
\expandafter\def\csname PY@tok@mf\endcsname{\def\PY@tc##1{\textcolor[rgb]{0.40,0.40,0.40}{##1}}}
\expandafter\def\csname PY@tok@mh\endcsname{\def\PY@tc##1{\textcolor[rgb]{0.40,0.40,0.40}{##1}}}
\expandafter\def\csname PY@tok@mi\endcsname{\def\PY@tc##1{\textcolor[rgb]{0.40,0.40,0.40}{##1}}}
\expandafter\def\csname PY@tok@il\endcsname{\def\PY@tc##1{\textcolor[rgb]{0.40,0.40,0.40}{##1}}}
\expandafter\def\csname PY@tok@mo\endcsname{\def\PY@tc##1{\textcolor[rgb]{0.40,0.40,0.40}{##1}}}
\expandafter\def\csname PY@tok@ch\endcsname{\let\PY@it=\textit\def\PY@tc##1{\textcolor[rgb]{0.25,0.50,0.50}{##1}}}
\expandafter\def\csname PY@tok@cm\endcsname{\let\PY@it=\textit\def\PY@tc##1{\textcolor[rgb]{0.25,0.50,0.50}{##1}}}
\expandafter\def\csname PY@tok@cpf\endcsname{\let\PY@it=\textit\def\PY@tc##1{\textcolor[rgb]{0.25,0.50,0.50}{##1}}}
\expandafter\def\csname PY@tok@c1\endcsname{\let\PY@it=\textit\def\PY@tc##1{\textcolor[rgb]{0.25,0.50,0.50}{##1}}}
\expandafter\def\csname PY@tok@cs\endcsname{\let\PY@it=\textit\def\PY@tc##1{\textcolor[rgb]{0.25,0.50,0.50}{##1}}}

\def\PYZbs{\char`\\}
\def\PYZus{\char`\_}
\def\PYZob{\char`\{}
\def\PYZcb{\char`\}}
\def\PYZca{\char`\^}
\def\PYZam{\char`\&}
\def\PYZlt{\char`\<}
\def\PYZgt{\char`\>}
\def\PYZsh{\char`\#}
\def\PYZpc{\char`\%}
\def\PYZdl{\char`\$}
\def\PYZhy{\char`\-}
\def\PYZsq{\char`\'}
\def\PYZdq{\char`\"}
\def\PYZti{\char`\~}
% for compatibility with earlier versions
\def\PYZat{@}
\def\PYZlb{[}
\def\PYZrb{]}
\makeatother


    % Exact colors from NB
    \definecolor{incolor}{rgb}{0.0, 0.0, 0.5}
    \definecolor{outcolor}{rgb}{0.545, 0.0, 0.0}



    
    % Prevent overflowing lines due to hard-to-break entities
    \sloppy 
    % Setup hyperref package
    \hypersetup{
      breaklinks=true,  % so long urls are correctly broken across lines
      colorlinks=true,
      urlcolor=urlcolor,
      linkcolor=linkcolor,
      citecolor=citecolor,
      }
    % Slightly bigger margins than the latex defaults
    
    \geometry{verbose,tmargin=1in,bmargin=1in,lmargin=1in,rmargin=1in}
    
    

    \begin{document}
    
    
    \maketitle
    
    

    
    \begin{Verbatim}[commandchars=\\\{\}]
{\color{incolor}In [{\color{incolor}1}]:} \PY{o}{!}pip install \PYZhy{}U okpy  \PYZsh{} Uncomment \PY{k}{if} you have an error
        \PY{k+kn}{from} \PY{n+nn}{client}\PY{n+nn}{.}\PY{n+nn}{api}\PY{n+nn}{.}\PY{n+nn}{notebook} \PY{k}{import} \PY{n}{Notebook}
        \PY{n}{ok} \PY{o}{=} \PY{n}{Notebook}\PY{p}{(}\PY{l+s+s1}{\PYZsq{}}\PY{l+s+s1}{hw1.ok}\PY{l+s+s1}{\PYZsq{}}\PY{p}{)}
\end{Verbatim}


    \begin{Verbatim}[commandchars=\\\{\}]
Collecting okpy
  Using cached okpy-1.18.1-py3-none-any.whl (107 kB)
Collecting certifi==2019.11.28
  Using cached certifi-2019.11.28-py2.py3-none-any.whl (156 kB)
Collecting filelock==3.0.12
  Using cached filelock-3.0.12-py3-none-any.whl (7.6 kB)
Requirement already satisfied: chardet==3.0.4 in /Users/sjk/anaconda3/envs/data100/lib/python3.6/site-packages (from okpy) (3.0.4)
Collecting ast-scope==0.3.1
  Using cached ast\_scope-0.3.1-py3-none-any.whl
Collecting requests==2.22.0
  Using cached requests-2.22.0-py2.py3-none-any.whl (57 kB)
Collecting coverage==4.4
  Using cached coverage-4.4-cp36-cp36m-macosx\_10\_10\_x86\_64.whl (172 kB)
Collecting pytutor==1.0.0
  Using cached pytutor-1.0.0-py3-none-any.whl
Collecting colorama==0.4.3
  Using cached colorama-0.4.3-py2.py3-none-any.whl (15 kB)
Collecting display-timedelta==1.1
  Using cached display\_timedelta-1.1-py3-none-any.whl
Collecting attrs==19.3.0
  Using cached attrs-19.3.0-py2.py3-none-any.whl (39 kB)
Collecting pyaes==1.6.1
  Using cached pyaes-1.6.1-py3-none-any.whl
Collecting urllib3==1.25.7
  Using cached urllib3-1.25.7-py2.py3-none-any.whl (125 kB)
Collecting idna==2.8
  Using cached idna-2.8-py2.py3-none-any.whl (58 kB)
\textcolor{ansi-yellow}{WARNING: Error parsing requirements for urllib3: [Errno 2] No such file or directory: '/Users/sjk/anaconda3/envs/data100/lib/python3.6/site-packages/urllib3-1.26.4.dist-info/METADATA'}
Installing collected packages: urllib3, idna, certifi, attrs, requests, pytutor, pyaes, filelock, display-timedelta, coverage, colorama, ast-scope, okpy
  Attempting uninstall: urllib3
    Found existing installation: urllib3 1.26.4
\textcolor{ansi-red}{ERROR: Could not install packages due to an OSError: [Errno 2] No such file or directory: '/Users/sjk/anaconda3/envs/data100/lib/python3.6/site-packages/urllib3-1.26.4.dist-info/RECORD'
}

    \end{Verbatim}

    \section{HW 1: Math Review and
Plotting}\label{hw-1-math-review-and-plotting}

\subsection{Due Date: Fri 3/26, 11:59
PM}\label{due-date-fri-326-1159-pm}

    \textbf{Collaboration Policy:} You may talk with others about the
homework, but we ask that you \textbf{write your solutions
individually}. If you do discuss the assignments with others, please
\textbf{include their names} in the following line.

\textbf{Collaborators}: \emph{list collaborators here (if applicable)}

    \subsection{This Assignment}\label{this-assignment}

This homework is to help you diagnose your preparedness for the course.
The rest of this course will assume familiarity with the programming and
math concepts covered in this homework. Please consider reviewing
prerequisite material if you struggle with this homework.

\subsection{Score Breakdown}\label{score-breakdown}

\begin{longtable}[]{@{}ll@{}}
\toprule
Question & Points\tabularnewline
\midrule
\endhead
1 & 1\tabularnewline
2a & 1\tabularnewline
2b & 1\tabularnewline
2c & 1\tabularnewline
2d & 1\tabularnewline
3a & 4\tabularnewline
3b & 2\tabularnewline
4a & 2\tabularnewline
4b & 2\tabularnewline
5 & 2\tabularnewline
6a & 2\tabularnewline
6b & 1\tabularnewline
6c & 1\tabularnewline
7 & 5\tabularnewline
Total & 30\tabularnewline
\bottomrule
\end{longtable}

    Here are some useful Jupyter notebook keyboard shortcuts. To learn more
keyboard shortcuts, go to \textbf{Help -\textgreater{} Keyboard
Shortcuts} in the menu above.

Here are a few we like: 1. \texttt{ctrl}+\texttt{return} :
\emph{Evaluate the current cell} 1. \texttt{shift}+\texttt{return}:
\emph{Evaluate the current cell and move to the next} 1. \texttt{esc} :
\emph{command mode} (may need to press before using any of the commands
below) 1. \texttt{a} : \emph{create a cell above} 1. \texttt{b} :
\emph{create a cell below} 1. \texttt{dd} : \emph{delete a cell} 1.
\texttt{m} : \emph{convert a cell to markdown} 1. \texttt{y} :
\emph{convert a cell to code}

    \subsubsection{Initialize your
environment}\label{initialize-your-environment}

This cell should run without error if you have \textbf{set up your
personal computer correctly}.

    \begin{Verbatim}[commandchars=\\\{\}]
{\color{incolor}In [{\color{incolor}2}]:} \PY{k+kn}{import} \PY{n+nn}{numpy} \PY{k}{as} \PY{n+nn}{np}
        \PY{k+kn}{import} \PY{n+nn}{matplotlib}
        \PY{o}{\PYZpc{}}\PY{k}{matplotlib} inline
        \PY{k+kn}{import} \PY{n+nn}{matplotlib}\PY{n+nn}{.}\PY{n+nn}{pyplot} \PY{k}{as} \PY{n+nn}{plt}
        \PY{n}{plt}\PY{o}{.}\PY{n}{style}\PY{o}{.}\PY{n}{use}\PY{p}{(}\PY{l+s+s1}{\PYZsq{}}\PY{l+s+s1}{fivethirtyeight}\PY{l+s+s1}{\PYZsq{}}\PY{p}{)}
        
        \PY{k+kn}{from} \PY{n+nn}{IPython}\PY{n+nn}{.}\PY{n+nn}{display} \PY{k}{import} \PY{n}{display}\PY{p}{,} \PY{n}{Latex}\PY{p}{,} \PY{n}{Markdown}
\end{Verbatim}


    \subsection{Python}\label{python}

\subsubsection{Question 1 (1 pt)}\label{question-1-1-pt}

Recall the formula for population variance below:

\[\texttt{Mean}({\bf{x}}) = \mu = \frac{1}{N}\sum_{i=1}^N x_i\]

\[\texttt{Var}({\bf{x}}) = \sigma^2 = \frac{1}{N}\sum_{i=1}^N (x_i - \mu)^2\]

Complete the functions below to compute the variance of
\texttt{population}, an array of numbers. For this question, do not use
built-in NumPy functions (i.e. \texttt{np.mean} and \texttt{np.var});
instead we will use NumPy to verify your code.

    \begin{Verbatim}[commandchars=\\\{\}]
{\color{incolor}In [{\color{incolor}7}]:} \PY{k}{def} \PY{n+nf}{mean}\PY{p}{(}\PY{n}{population}\PY{p}{)}\PY{p}{:}
            \PY{l+s+sd}{\PYZdq{}\PYZdq{}\PYZdq{}}
        \PY{l+s+sd}{    Compute the mean of population (mu).}
        \PY{l+s+sd}{    }
        \PY{l+s+sd}{    Args:}
        \PY{l+s+sd}{        population: a numpy array of numbers of shape [N,]}
        \PY{l+s+sd}{    Returns:}
        \PY{l+s+sd}{        the mean of population (mu).}
        \PY{l+s+sd}{    \PYZdq{}\PYZdq{}\PYZdq{}}
            \PY{c+c1}{\PYZsh{} Calculate the mean of a population}
            \PY{c+c1}{\PYZsh{} BEGIN YOUR CODE}
            \PY{c+c1}{\PYZsh{} \PYZhy{}\PYZhy{}\PYZhy{}\PYZhy{}\PYZhy{}\PYZhy{}\PYZhy{}\PYZhy{}\PYZhy{}\PYZhy{}\PYZhy{}\PYZhy{}\PYZhy{}\PYZhy{}\PYZhy{}\PYZhy{}\PYZhy{}\PYZhy{}\PYZhy{}\PYZhy{}\PYZhy{}\PYZhy{}\PYZhy{}}
            \PY{n}{total}\PY{o}{=}\PY{l+m+mi}{0}
            \PY{k}{for} \PY{n}{N} \PY{o+ow}{in} \PY{n}{population}\PY{p}{:}
                \PY{n}{total} \PY{o}{+}\PY{o}{=} \PY{n}{N}
            \PY{k}{return} \PY{n}{total}\PY{o}{/}\PY{n+nb}{len}\PY{p}{(}\PY{n}{population}\PY{p}{)}
        
            \PY{c+c1}{\PYZsh{} \PYZhy{}\PYZhy{}\PYZhy{}\PYZhy{}\PYZhy{}\PYZhy{}\PYZhy{}\PYZhy{}\PYZhy{}\PYZhy{}\PYZhy{}\PYZhy{}\PYZhy{}\PYZhy{}\PYZhy{}\PYZhy{}\PYZhy{}\PYZhy{}\PYZhy{}\PYZhy{}\PYZhy{}\PYZhy{}\PYZhy{}}
            \PY{c+c1}{\PYZsh{} END YOUR CODE}
        
        \PY{k}{def} \PY{n+nf}{variance}\PY{p}{(}\PY{n}{population}\PY{p}{)}\PY{p}{:}
            \PY{l+s+sd}{\PYZdq{}\PYZdq{}\PYZdq{}}
        \PY{l+s+sd}{    Compute the variance of population (sigma squared).}
        \PY{l+s+sd}{    }
        \PY{l+s+sd}{    Args:}
        \PY{l+s+sd}{        population: a numpy array of numbers of shape [N,]}
        \PY{l+s+sd}{    Returns:}
        \PY{l+s+sd}{        the variance of population}
        \PY{l+s+sd}{    \PYZdq{}\PYZdq{}\PYZdq{}}
            \PY{c+c1}{\PYZsh{} Calculate the variance of a population}
            \PY{c+c1}{\PYZsh{} BEGIN YOUR CODE}
            \PY{c+c1}{\PYZsh{} \PYZhy{}\PYZhy{}\PYZhy{}\PYZhy{}\PYZhy{}\PYZhy{}\PYZhy{}\PYZhy{}\PYZhy{}\PYZhy{}\PYZhy{}\PYZhy{}\PYZhy{}\PYZhy{}\PYZhy{}\PYZhy{}\PYZhy{}\PYZhy{}\PYZhy{}\PYZhy{}\PYZhy{}\PYZhy{}\PYZhy{}}
            \PY{n}{total} \PY{o}{=} \PY{l+m+mi}{0}
            \PY{n}{m} \PY{o}{=} \PY{n}{mean}\PY{p}{(}\PY{n}{population}\PY{p}{)}
            \PY{k}{for} \PY{n}{N} \PY{o+ow}{in} \PY{n}{population}\PY{p}{:}
                \PY{n}{total} \PY{o}{+}\PY{o}{=} \PY{p}{(}\PY{n}{m} \PY{o}{\PYZhy{}} \PY{n}{N}\PY{p}{)}\PY{o}{*}\PY{o}{*}\PY{l+m+mi}{2}
            \PY{k}{return} \PY{n}{total}\PY{o}{/}\PY{n+nb}{len}\PY{p}{(}\PY{n}{population}\PY{p}{)}
            \PY{c+c1}{\PYZsh{} \PYZhy{}\PYZhy{}\PYZhy{}\PYZhy{}\PYZhy{}\PYZhy{}\PYZhy{}\PYZhy{}\PYZhy{}\PYZhy{}\PYZhy{}\PYZhy{}\PYZhy{}\PYZhy{}\PYZhy{}\PYZhy{}\PYZhy{}\PYZhy{}\PYZhy{}\PYZhy{}\PYZhy{}\PYZhy{}\PYZhy{}}
            \PY{c+c1}{\PYZsh{} END YOUR CODE}
\end{Verbatim}


    \begin{Verbatim}[commandchars=\\\{\}]
{\color{incolor}In [{\color{incolor}8}]:} \PY{n}{ok}\PY{o}{.}\PY{n}{grade}\PY{p}{(}\PY{l+s+s2}{\PYZdq{}}\PY{l+s+s2}{q1}\PY{l+s+s2}{\PYZdq{}}\PY{p}{)}\PY{p}{;}
\end{Verbatim}


    
    \begin{verbatim}
<okgrade.result.TestResult at 0x10f0a7b38>
    \end{verbatim}

    
    \begin{center}\rule{0.5\linewidth}{\linethickness}\end{center}

\subsection{NumPy}\label{numpy}

You should be able to understand the code in the following cells. If
not, please review the following:

\begin{itemize}
\tightlist
\item
  \href{http://ds100.org/fa17/assets/notebooks/numpy/Numpy_Review.html}{UC
  Berkeley DS100 NumPy Review}
\item
  \href{http://cs231n.github.io/python-numpy-tutorial/\#numpy}{Stanford
  Condensed NumPy Review}
\item
  \href{https://docs.scipy.org/doc/numpy-dev/user/quickstart.html}{The
  Official NumPy Tutorial}
\item
  \href{https://www.inferentialthinking.com/chapters/05/1/Arrays}{UC
  Berkeley Data 8 Textbook Chapter on NumPy}
\end{itemize}

    \textbf{Jupyter pro-tip}: Pull up the docs for any function in Jupyter
by running a cell with the function name and a \texttt{?} at the end:

    \begin{Verbatim}[commandchars=\\\{\}]
{\color{incolor}In [{\color{incolor}9}]:} np.arange\PY{o}{?}
\end{Verbatim}


    You can close the window at the bottom by pressing \texttt{esc} several
times.

    \textbf{Another Jupyter pro-tip}: Pull up the docs for any function in
Jupyter by typing the function name, then
\texttt{\textless{}Shift\textgreater{}-\textless{}Tab\textgreater{}} on
your keyboard. This is super convenient when you forget the order of the
arguments to a function. You can press
\texttt{\textless{}Tab\textgreater{}} multiple times to expand the docs
and reveal additional information.

Try it on the function below:

    \begin{Verbatim}[commandchars=\\\{\}]
{\color{incolor}In [{\color{incolor}11}]:} \PY{n}{np}\PY{o}{.}\PY{n}{linspace}
\end{Verbatim}


    Now, let's go through some linear algebra coding questions with NumPy.
In this question, we'll ask you to use your linear algebra knowledge to
fill in NumPy matrices. To conduct matrix multiplication in NumPy, you
should write code like the following:

    \begin{Verbatim}[commandchars=\\\{\}]
{\color{incolor}In [{\color{incolor}12}]:} \PY{c+c1}{\PYZsh{} A matrix in NumPy is a 2\PYZhy{}dimensional NumPy array}
         \PY{n}{matA} \PY{o}{=} \PY{n}{np}\PY{o}{.}\PY{n}{array}\PY{p}{(}\PY{p}{[}
             \PY{p}{[}\PY{l+m+mi}{1}\PY{p}{,} \PY{l+m+mi}{2}\PY{p}{,} \PY{l+m+mi}{3}\PY{p}{]}\PY{p}{,}
             \PY{p}{[}\PY{l+m+mi}{4}\PY{p}{,} \PY{l+m+mi}{5}\PY{p}{,} \PY{l+m+mi}{6}\PY{p}{]}\PY{p}{,}
         \PY{p}{]}\PY{p}{)}
         
         \PY{n}{matB} \PY{o}{=} \PY{n}{np}\PY{o}{.}\PY{n}{array}\PY{p}{(}\PY{p}{[}
             \PY{p}{[}\PY{l+m+mi}{10}\PY{p}{,} \PY{l+m+mi}{11}\PY{p}{]}\PY{p}{,}
             \PY{p}{[}\PY{l+m+mi}{12}\PY{p}{,} \PY{l+m+mi}{13}\PY{p}{]}\PY{p}{,}
             \PY{p}{[}\PY{l+m+mi}{14}\PY{p}{,} \PY{l+m+mi}{15}\PY{p}{]}\PY{p}{,}
         \PY{p}{]}\PY{p}{)}
         
         \PY{c+c1}{\PYZsh{} The notation B @ v means: compute the matrix multiplication Bv}
         \PY{n}{matA} \PY{o}{@} \PY{n}{matB}
\end{Verbatim}


\begin{Verbatim}[commandchars=\\\{\}]
{\color{outcolor}Out[{\color{outcolor}12}]:} array([[ 76,  82],
                [184, 199]])
\end{Verbatim}
            
    You can also use the same syntax to do matrix-vector multiplication or
vector dot products. Handy!

    \begin{Verbatim}[commandchars=\\\{\}]
{\color{incolor}In [{\color{incolor}13}]:} \PY{n}{matA} \PY{o}{=} \PY{n}{np}\PY{o}{.}\PY{n}{array}\PY{p}{(}\PY{p}{[}
             \PY{p}{[}\PY{l+m+mi}{1}\PY{p}{,} \PY{l+m+mi}{2}\PY{p}{,} \PY{l+m+mi}{3}\PY{p}{]}\PY{p}{,}
             \PY{p}{[}\PY{l+m+mi}{4}\PY{p}{,} \PY{l+m+mi}{5}\PY{p}{,} \PY{l+m+mi}{6}\PY{p}{]}\PY{p}{,}
         \PY{p}{]}\PY{p}{)}
         
         \PY{c+c1}{\PYZsh{} A vector in NumPy is simply a 1\PYZhy{}dimensional NumPy array}
         \PY{n}{some\PYZus{}vec} \PY{o}{=} \PY{n}{np}\PY{o}{.}\PY{n}{array}\PY{p}{(}\PY{p}{[} \PY{l+m+mi}{10}\PY{p}{,} \PY{l+m+mi}{12}\PY{p}{,} \PY{l+m+mi}{14}\PY{p}{,} \PY{p}{]}\PY{p}{)}
         \PY{n}{another\PYZus{}vec} \PY{o}{=} \PY{n}{np}\PY{o}{.}\PY{n}{array}\PY{p}{(}\PY{p}{[} \PY{l+m+mi}{10}\PY{p}{,} \PY{l+m+mi}{20}\PY{p}{,} \PY{l+m+mi}{30} \PY{p}{]}\PY{p}{)}
         
         \PY{n+nb}{print}\PY{p}{(}\PY{n}{matA} \PY{o}{@} \PY{n}{some\PYZus{}vec}\PY{p}{)}
         \PY{n+nb}{print}\PY{p}{(}\PY{n}{some\PYZus{}vec} \PY{o}{@} \PY{n}{another\PYZus{}vec}\PY{p}{)}
\end{Verbatim}


    \begin{Verbatim}[commandchars=\\\{\}]
[ 76 184]
760

    \end{Verbatim}

    \subsection{Question 2 (4 pt)}\label{question-2-4-pt}

\subsubsection{Question 2a}\label{question-2a}

Joey, Deb, and Sam are shopping for fruit at K-Bowl. K-Bowl, true to its
name, only sells fruit bowls. A fruit bowl contains some fruit and the
price of a fruit bowl is the total price of all of its individual fruit.

Berkeley Bowl has apples for \textbackslash{}\$2.00, bananas for
\textbackslash{}\$1.00, and cantaloupes for \textbackslash{}\$4.00
(expensive!). The price of each of these can be written in a vector:

\[
v = \begin{bmatrix}
     2 \\
     1 \\
     4 \\
\end{bmatrix}
\]

K-Bowl sells the following fruit bowls:

\begin{enumerate}
\def\labelenumi{\arabic{enumi}.}
\tightlist
\item
  2 of each fruit
\item
  5 apples and 8 bananas
\item
  2 bananas and 3 cantaloupes
\item
  10 cantaloupes
\end{enumerate}

Create a 2-dimensional numpy array encoding the matrix \(B\) such that
the matrix-vector multiplication

\[
Bv
\]

evaluates to a length 4 column vector containing the price of each fruit
bowl. The first entry of the result should be the cost of fruit bowl
\#1, the second entry the cost of fruit bowl \#2, etc.

    \begin{Verbatim}[commandchars=\\\{\}]
{\color{incolor}In [{\color{incolor}18}]:} \PY{n}{v} \PY{o}{=} \PY{n}{np}\PY{o}{.}\PY{n}{array}\PY{p}{(}\PY{p}{[}\PY{l+m+mi}{2}\PY{p}{,}\PY{l+m+mi}{1}\PY{p}{,}\PY{l+m+mi}{4}\PY{p}{]}\PY{p}{)}
         
         \PY{c+c1}{\PYZsh{} BEGIN YOUR CODE}
         \PY{c+c1}{\PYZsh{} \PYZhy{}\PYZhy{}\PYZhy{}\PYZhy{}\PYZhy{}\PYZhy{}\PYZhy{}\PYZhy{}\PYZhy{}\PYZhy{}\PYZhy{}\PYZhy{}\PYZhy{}\PYZhy{}\PYZhy{}\PYZhy{}\PYZhy{}\PYZhy{}\PYZhy{}\PYZhy{}\PYZhy{}\PYZhy{}\PYZhy{}}
         \PY{n}{B} \PY{o}{=} \PY{n}{np}\PY{o}{.}\PY{n}{array}\PY{p}{(}\PY{p}{[}
             \PY{p}{[}\PY{l+m+mi}{2}\PY{p}{,} \PY{l+m+mi}{2}\PY{p}{,} \PY{l+m+mi}{2}\PY{p}{]}\PY{p}{,}
             \PY{p}{[}\PY{l+m+mi}{5}\PY{p}{,} \PY{l+m+mi}{8}\PY{p}{,} \PY{l+m+mi}{0}\PY{p}{]}\PY{p}{,}
             \PY{p}{[}\PY{l+m+mi}{0}\PY{p}{,} \PY{l+m+mi}{2}\PY{p}{,} \PY{l+m+mi}{3}\PY{p}{]}\PY{p}{,}
             \PY{p}{[}\PY{l+m+mi}{0}\PY{p}{,} \PY{l+m+mi}{0}\PY{p}{,}\PY{l+m+mi}{10}\PY{p}{]}
         \PY{p}{]}\PY{p}{)}
         \PY{c+c1}{\PYZsh{} \PYZhy{}\PYZhy{}\PYZhy{}\PYZhy{}\PYZhy{}\PYZhy{}\PYZhy{}\PYZhy{}\PYZhy{}\PYZhy{}\PYZhy{}\PYZhy{}\PYZhy{}\PYZhy{}\PYZhy{}\PYZhy{}\PYZhy{}\PYZhy{}\PYZhy{}\PYZhy{}\PYZhy{}\PYZhy{}\PYZhy{}}
         \PY{c+c1}{\PYZsh{} END YOUR CODE}
         
         \PY{c+c1}{\PYZsh{} The notation B @ v means: compute the matrix multiplication Bv}
         \PY{n}{B} \PY{o}{@} \PY{n}{v}
\end{Verbatim}


\begin{Verbatim}[commandchars=\\\{\}]
{\color{outcolor}Out[{\color{outcolor}18}]:} array([14, 18, 14, 40])
\end{Verbatim}
            
    \begin{Verbatim}[commandchars=\\\{\}]
{\color{incolor}In [{\color{incolor}19}]:} \PY{n}{ok}\PY{o}{.}\PY{n}{grade}\PY{p}{(}\PY{l+s+s2}{\PYZdq{}}\PY{l+s+s2}{q2a}\PY{l+s+s2}{\PYZdq{}}\PY{p}{)}\PY{p}{;}
\end{Verbatim}


    
    \begin{verbatim}
<okgrade.result.TestResult at 0x1125de358>
    \end{verbatim}

    
    \subsubsection{Question 2b}\label{question-2b}

Joey, Deb, and Sam make the following purchases:

\begin{itemize}
\tightlist
\item
  Joey buys 2 fruit bowl \#1s and 1 fruit bowl \#2.
\item
  Deb buys 1 of each fruit bowl.
\item
  Sam buys 10 fruit bowl \#4s (he really like cantaloupes).
\end{itemize}

Create a matrix \(A\) such that the matrix expression

\[
ABv
\]

evaluates to a length 3 column vector containing how much each of them
spent. The first entry of the result should be the total amount spent by
Joey, the second entry the amount sent by Deb, etc.

Note that the tests for this question do not tell you whether your
answer is correct. That's up to you to determine.

    \begin{Verbatim}[commandchars=\\\{\}]
{\color{incolor}In [{\color{incolor}20}]:} \PY{n}{A} \PY{o}{=} \PY{n}{np}\PY{o}{.}\PY{n}{array}\PY{p}{(}\PY{p}{[}
             \PY{p}{[}\PY{l+m+mi}{2}\PY{p}{,} \PY{l+m+mi}{1}\PY{p}{,} \PY{l+m+mi}{0}\PY{p}{,} \PY{l+m+mi}{0}\PY{p}{]}\PY{p}{,}
             \PY{c+c1}{\PYZsh{} Finish this!}
             \PY{c+c1}{\PYZsh{} BEGIN YOUR CODE}
             \PY{c+c1}{\PYZsh{} \PYZhy{}\PYZhy{}\PYZhy{}\PYZhy{}\PYZhy{}\PYZhy{}\PYZhy{}\PYZhy{}\PYZhy{}\PYZhy{}\PYZhy{}\PYZhy{}\PYZhy{}\PYZhy{}\PYZhy{}\PYZhy{}\PYZhy{}\PYZhy{}\PYZhy{}\PYZhy{}\PYZhy{}\PYZhy{}\PYZhy{}}
             \PY{p}{[}\PY{l+m+mi}{1}\PY{p}{,} \PY{l+m+mi}{1}\PY{p}{,} \PY{l+m+mi}{1}\PY{p}{,} \PY{l+m+mi}{1}\PY{p}{]}\PY{p}{,}
             \PY{p}{[}\PY{l+m+mi}{0}\PY{p}{,} \PY{l+m+mi}{0}\PY{p}{,} \PY{l+m+mi}{0}\PY{p}{,}\PY{l+m+mi}{10}\PY{p}{]}
             \PY{c+c1}{\PYZsh{} \PYZhy{}\PYZhy{}\PYZhy{}\PYZhy{}\PYZhy{}\PYZhy{}\PYZhy{}\PYZhy{}\PYZhy{}\PYZhy{}\PYZhy{}\PYZhy{}\PYZhy{}\PYZhy{}\PYZhy{}\PYZhy{}\PYZhy{}\PYZhy{}\PYZhy{}\PYZhy{}\PYZhy{}\PYZhy{}\PYZhy{}}
             \PY{c+c1}{\PYZsh{} END YOUR CODE}
         \PY{p}{]}\PY{p}{)} 
         
         \PY{n}{A} \PY{o}{@} \PY{n}{B} \PY{o}{@} \PY{n}{v} 
\end{Verbatim}


\begin{Verbatim}[commandchars=\\\{\}]
{\color{outcolor}Out[{\color{outcolor}20}]:} array([ 46,  86, 400])
\end{Verbatim}
            
    \begin{Verbatim}[commandchars=\\\{\}]
{\color{incolor}In [{\color{incolor}21}]:} \PY{n}{ok}\PY{o}{.}\PY{n}{grade}\PY{p}{(}\PY{l+s+s2}{\PYZdq{}}\PY{l+s+s2}{q2b}\PY{l+s+s2}{\PYZdq{}}\PY{p}{)}\PY{p}{;}
\end{Verbatim}


    
    \begin{verbatim}
<okgrade.result.TestResult at 0x1125d9588>
    \end{verbatim}

    
    \subsubsection{Question 2c}\label{question-2c}

Who spent the most money? Assign \texttt{most} to a string containing
the name of this person.

    \begin{Verbatim}[commandchars=\\\{\}]
{\color{incolor}In [{\color{incolor}34}]:} \PY{c+c1}{\PYZsh{} BEGIN YOUR CODE}
         \PY{c+c1}{\PYZsh{} \PYZhy{}\PYZhy{}\PYZhy{}\PYZhy{}\PYZhy{}\PYZhy{}\PYZhy{}\PYZhy{}\PYZhy{}\PYZhy{}\PYZhy{}\PYZhy{}\PYZhy{}\PYZhy{}\PYZhy{}\PYZhy{}\PYZhy{}\PYZhy{}\PYZhy{}\PYZhy{}\PYZhy{}\PYZhy{}\PYZhy{}}
         \PY{n}{most} \PY{o}{=} \PY{l+s+s2}{\PYZdq{}}\PY{l+s+s2}{Sam}\PY{l+s+s2}{\PYZdq{}}
         \PY{c+c1}{\PYZsh{} \PYZhy{}\PYZhy{}\PYZhy{}\PYZhy{}\PYZhy{}\PYZhy{}\PYZhy{}\PYZhy{}\PYZhy{}\PYZhy{}\PYZhy{}\PYZhy{}\PYZhy{}\PYZhy{}\PYZhy{}\PYZhy{}\PYZhy{}\PYZhy{}\PYZhy{}\PYZhy{}\PYZhy{}\PYZhy{}\PYZhy{}}
         \PY{c+c1}{\PYZsh{} END YOUR CODE}
\end{Verbatim}


    \begin{Verbatim}[commandchars=\\\{\}]
{\color{incolor}In [{\color{incolor}35}]:} \PY{n}{ok}\PY{o}{.}\PY{n}{grade}\PY{p}{(}\PY{l+s+s2}{\PYZdq{}}\PY{l+s+s2}{q2c}\PY{l+s+s2}{\PYZdq{}}\PY{p}{)}\PY{p}{;}
\end{Verbatim}


    
    \begin{verbatim}
<okgrade.result.TestResult at 0x1125de978>
    \end{verbatim}

    
    \subsubsection{Question 2d}\label{question-2d}

Let's suppose K-Bowl changes their fruit prices, but you don't know what
they changed their prices to. Joey, Deb, and Sam buy the same quantity
of fruit baskets and the number of fruit in each basket is the same, but
now they each spent these amounts:

\[
x = \begin{bmatrix}
    80 \\
    80 \\
    100 \\
\end{bmatrix}
\]

Use \texttt{np.linalg.inv} and the above final costs to compute the new
prices for the individual fruits as a vector called \texttt{new\_v}.

    \begin{Verbatim}[commandchars=\\\{\}]
{\color{incolor}In [{\color{incolor}56}]:} \PY{c+c1}{\PYZsh{} BEGIN YOUR CODE}
         \PY{c+c1}{\PYZsh{} \PYZhy{}\PYZhy{}\PYZhy{}\PYZhy{}\PYZhy{}\PYZhy{}\PYZhy{}\PYZhy{}\PYZhy{}\PYZhy{}\PYZhy{}\PYZhy{}\PYZhy{}\PYZhy{}\PYZhy{}\PYZhy{}\PYZhy{}\PYZhy{}\PYZhy{}\PYZhy{}\PYZhy{}\PYZhy{}\PYZhy{}}
         \PY{n}{x} \PY{o}{=} \PY{n}{np}\PY{o}{.}\PY{n}{array}\PY{p}{(}\PY{p}{[}\PY{l+m+mi}{80}\PY{p}{,} \PY{l+m+mi}{80}\PY{p}{,} \PY{l+m+mi}{100}\PY{p}{]}\PY{p}{)}
         \PY{n}{new\PYZus{}v} \PY{o}{=} \PY{n}{np}\PY{o}{.}\PY{n}{linalg}\PY{o}{.}\PY{n}{inv}\PY{p}{(}\PY{n}{A} \PY{o}{@} \PY{n}{B}\PY{p}{)} \PY{o}{@} \PY{n}{x}
         \PY{c+c1}{\PYZsh{} \PYZhy{}\PYZhy{}\PYZhy{}\PYZhy{}\PYZhy{}\PYZhy{}\PYZhy{}\PYZhy{}\PYZhy{}\PYZhy{}\PYZhy{}\PYZhy{}\PYZhy{}\PYZhy{}\PYZhy{}\PYZhy{}\PYZhy{}\PYZhy{}\PYZhy{}\PYZhy{}\PYZhy{}\PYZhy{}\PYZhy{}}
         \PY{c+c1}{\PYZsh{} END YOUR CODE}
         \PY{n}{new\PYZus{}v}
\end{Verbatim}


\begin{Verbatim}[commandchars=\\\{\}]
{\color{outcolor}Out[{\color{outcolor}56}]:} array([ 5.5       ,  2.20833333,  1.        ])
\end{Verbatim}
            
    \begin{Verbatim}[commandchars=\\\{\}]
{\color{incolor}In [{\color{incolor}57}]:} \PY{n}{tmp} \PY{o}{=} \PY{n}{ok}\PY{o}{.}\PY{n}{grade}\PY{p}{(}\PY{l+s+s2}{\PYZdq{}}\PY{l+s+s2}{q2d}\PY{l+s+s2}{\PYZdq{}}\PY{p}{)}\PY{p}{;}
\end{Verbatim}


    
    \begin{verbatim}
<okgrade.result.TestResult at 0x1125d9b38>
    \end{verbatim}

    
    \begin{center}\rule{0.5\linewidth}{\linethickness}\end{center}

\subsection{Multivariable Calculus, Linear Algebra, and
Probability}\label{multivariable-calculus-linear-algebra-and-probability}

The following questions ask you to recall your knowledge of
multivariable calculus, linear algebra, and probability. We will use
some of the most fundamental concepts from each discipline in this
class, so the following problems should at least seem familiar to you.

If you have trouble with these topics, we suggest reviewing:

\begin{itemize}
\tightlist
\item
  \href{https://www.khanacademy.org/math/multivariable-calculus}{Khan
  Academy's Multivariable Calculus}
\item
  \href{https://www.khanacademy.org/math/linear-algebra}{Khan Academy's
  Linear Algebra}
\item
  \href{https://www.khanacademy.org/math/statistics-probability}{Khan
  Academy's Statistics and Probability}
\end{itemize}

\paragraph{LaTeX}\label{latex}

For the following problems, you should use LaTeX to format your answer.
If you aren't familiar with LaTeX, not to worry. It's not hard to use in
a Jupyter notebook. Just place your math in between dollar signs:

\textbackslash{}\$ f(x) = 2x \textbackslash{}\$ becomes \$ f(x) = 2x \$.

If you have a longer equation, use double dollar signs to place it on a
line by itself:

\textbackslash{}\(\\\) \sum\_\{i=0\}\^{}n i\^{}2 \textbackslash{}\(\\\)
becomes:

\[ \sum_{i=0}^n i^2 \].

Here is some handy notation:

\begin{longtable}[]{@{}ll@{}}
\toprule
\begin{minipage}[b]{0.07\columnwidth}\raggedright\strut
Output\strut
\end{minipage} & \begin{minipage}[b]{0.07\columnwidth}\raggedright\strut
Latex\strut
\end{minipage}\tabularnewline
\midrule
\endhead
\begin{minipage}[t]{0.07\columnwidth}\raggedright\strut
\[x^{a + b}\]\strut
\end{minipage} & \begin{minipage}[t]{0.07\columnwidth}\raggedright\strut
\texttt{x\^{}\{a\ +\ b\}}\strut
\end{minipage}\tabularnewline
\begin{minipage}[t]{0.07\columnwidth}\raggedright\strut
\[x_{a + b}\]\strut
\end{minipage} & \begin{minipage}[t]{0.07\columnwidth}\raggedright\strut
\texttt{x\_\{a\ +\ b\}}\strut
\end{minipage}\tabularnewline
\begin{minipage}[t]{0.07\columnwidth}\raggedright\strut
\[\frac{a}{b}\]\strut
\end{minipage} & \begin{minipage}[t]{0.07\columnwidth}\raggedright\strut
\texttt{\textbackslash{}frac\{a\}\{b\}}\strut
\end{minipage}\tabularnewline
\begin{minipage}[t]{0.07\columnwidth}\raggedright\strut
\[\sqrt{a + b}\]\strut
\end{minipage} & \begin{minipage}[t]{0.07\columnwidth}\raggedright\strut
\texttt{\textbackslash{}sqrt\{a\ +\ b\}}\strut
\end{minipage}\tabularnewline
\begin{minipage}[t]{0.07\columnwidth}\raggedright\strut
\[\{ \alpha, \beta, \gamma, \pi, \mu, \sigma^2  \}\]\strut
\end{minipage} & \begin{minipage}[t]{0.07\columnwidth}\raggedright\strut
\texttt{\textbackslash{}\{\ \textbackslash{}alpha,\ \textbackslash{}beta,\ \textbackslash{}gamma,\ \textbackslash{}pi,\ \textbackslash{}mu,\ \textbackslash{}sigma\^{}2\ \ \textbackslash{}\}}\strut
\end{minipage}\tabularnewline
\begin{minipage}[t]{0.07\columnwidth}\raggedright\strut
\[\sum_{x=1}^{100}\]\strut
\end{minipage} & \begin{minipage}[t]{0.07\columnwidth}\raggedright\strut
\texttt{\textbackslash{}sum\_\{x=1\}\^{}\{100\}}\strut
\end{minipage}\tabularnewline
\begin{minipage}[t]{0.07\columnwidth}\raggedright\strut
\[\frac{\partial}{\partial x} \]\strut
\end{minipage} & \begin{minipage}[t]{0.07\columnwidth}\raggedright\strut
\texttt{\textbackslash{}frac\{\textbackslash{}partial\}\{\textbackslash{}partial\ x\}}\strut
\end{minipage}\tabularnewline
\begin{minipage}[t]{0.07\columnwidth}\raggedright\strut
\[\begin{bmatrix} 2x + 4y \\ 4x + 6y^2 \\ \end{bmatrix}\]\strut
\end{minipage} & \begin{minipage}[t]{0.07\columnwidth}\raggedright\strut
\texttt{\textbackslash{}begin\{bmatrix\}\ 2x\ +\ 4y\ \textbackslash{}\textbackslash{}\ 4x\ +\ 6y\^{}2\ \textbackslash{}\textbackslash{}\ \textbackslash{}end\{bmatrix\}}\strut
\end{minipage}\tabularnewline
\bottomrule
\end{longtable}

\href{https://www.sharelatex.com/learn/Mathematical_expressions}{For
more about basic LaTeX formatting, you can read this article.}

    \subsubsection{Question 3a (4 pt)}\label{question-3a-4-pt}

Suppose we have the following scalar-valued function:

\[ f(x, y) = x^2 + 4xy + 2y^3 + e^{-3y} + \ln(2y) \]

Compute the partial derivative \(\frac{\partial}{\partial x} f(x,y)\):

    Answer: \(2x + 4y\)

    Now compute the partial derivative
\(\frac{\partial}{\partial y} f(x,y)\):

    Answer: \(4x + 6y^2 + \frac{1}{y} - 3e^{-3y}\)

    Finally, using your answers to the above two parts, compute
\(\nabla f(x, y)\). Also what is the gradient at the point (x, y) = (2,
-1):

Note that \(\nabla\) represents the gradient.

    Answer:
\(\nabla f(x, y) = (2x + 4y)i + (4x + 6y^2 + \frac{1}{y} - 3e^{-3y})j = (2x + 4y, 4x + 6y^2 + \frac{1}{y} - 3e^{-3y})\)

\(\nabla f(2, -1) = (2(2) + 4(-1), 4(2) + 6(-1)^2 + \frac{1}{-1} - 3e^{-3(-1)}) = (4 - 4, 8 + 6 - 1 - 3e^3) = (0, 13-3e^3)\)

    \subsubsection{Question 3b (2 pt)}\label{question-3b-2-pt}

Find the value(s) of \(x\) which minimizes the expression below. Justify
why it is the minimum.

\(\sum_{i=1}^{10} (i - x)^2\)

    Answer:
\texttt{i\ can\ be\ 1,2,3,4,5,6,7,8,9,10.\ The\ median\ of\ these\ values\ is\ 5.5\ and\ the\ median\ will\ minimize\ the\ expression\ so\ x\ =\ 5.5}

Let \(f(x) = \sum_{i=1}^{10} (i - x)^2\)

\(f'(x) = -2\sum_{i=1}^{10} (i - x) = -2(55 - 10x) = 0\)

\(x = 5.5\)

    \subsubsection{Question 4a (2 pt)}\label{question-4a-2-pt}

Let \(\sigma(x) = \dfrac{1}{1+e^{-x}}\).

Show that \(\sigma(-x) = 1 - \sigma(x)\).

    Answer:

\(\sigma(-x) = \dfrac{1}{1+e^{-(-x)}} = \dfrac{1}{1+e^{x}}\)

\(1 - \sigma(x) = 1 - \dfrac{1}{1+e^{-x}} = \dfrac{1+e^{-x}-1}{1+e^{-x}} = \dfrac{e^{-x}}{1+e^{-x}} = \dfrac{\dfrac{1}{e^x}}{1+\dfrac{1}{e^x}} = \dfrac{\dfrac{1}{e^x}}{\dfrac{e^x+1}{e^x}} = \dfrac{1}{e^x+1}\)

\$ \sigma(-x) = 1 - \sigma(x) \$

    \subsubsection{Question 4b (2 pt)}\label{question-4b-2-pt}

Show that the derivative can be written as:

\[\frac{d}{dx}\sigma(x) = \sigma(x)(1 - \sigma(x))\]

    Answer:

\$\frac{d}{dx}\sigma(x) = \frac{d}{dx}(\dfrac{1}{1+e^{-x}}) =
\frac{d}{dx}(\dfrac{e^x}{e^x+1}) \$

\$= \dfrac{(e^x+1)e^x - e^x(e^x)}{(e^x+1)^2} \$

\$= \dfrac{e^2x+e^x - e^2x}{(e^x+1)^2} \$

\$= \dfrac{e^x}{(e^x+1)^2} \$

\$\sigma(x)(1 - \sigma(x)) = (\dfrac{1}{1+e^{-x}})(\dfrac{1}{1+e^{x}}) =
(\dfrac{e^x}{e^x+1})(\dfrac{1}{1+e^{x}}) = \dfrac{e^x}{(e^x+1)^2} \$

\(\frac{d}{dx}\sigma(x) = \sigma(x)(1 - \sigma(x))\)

    \subsubsection{Question 5 (2 pt)}\label{question-5-2-pt}

Consider the following scenario:

Only 1\% of 40-year-old women who participate in a routine mammography
test have breast cancer. 80\% of women who have breast cancer will test
positive, but 9.6\% of women who don't have breast cancer will also get
positive tests.

Suppose we know that a woman of this age tested positive in a routine
screening. What is the probability that she actually has breast cancer?

\textbf{Hint:} Use Bayes' rule.

    Answer: P(Has cancer) = 0.01

P(Test positive \textbar{} has cancer) = 0.8

P(Test positive \textbar{} has no cancer) = 0.096

P(Has cancer \textbar{} test positive) = ?

P(A\textbar{}B) = P(B\textbar{}A)P(A) / P(B)

P(Has cancer \textbar{} test positive) = P(Test positive \textbar{} has
cancer) x P(Has cancer) / P(Test positive)

= 0.8 x 0.01 / (0.8 x 0.01 + 0.096 x 0.99) = 0.8 x 0.01 / 0.10304

= 0.07763975155

= 7.8\%

    \subsubsection{Question 6}\label{question-6}

Consider (once again) a sample of size n drawn at random with
replacement from a population in which a proportion p of the individuals
are called successes.

Let S be the random variable that denotes the number of successes in our
sample. (As stated above, S follows the binomial distribution.) Then,
the probability that the number of successes in our sample is \textbf{at
most} s (where \(0 \leq s \leq n\)) is

\[P(S \leq s) = P(S = 0) + P(S = 1) + ... + P(S = s) = \sum_{k=0}^s \binom{n}{k}p^k(1-p)^{n-k}\]

We obtain this by summing the probability that the number of successes
is exactly k, for each value of \(k = 0, 1, 2, ..., s\).

    \subsubsection{Question 6a (2pt)}\label{question-6a-2pt}

Please fill in the function \texttt{prob\_at\_most} which takes n, p,
and s and returns \(P(S \le s)\) as defined above. If the inputs are
invalid: for instance, if p \textgreater{} 1 or s \textgreater{} n then
return 0."

\textbf{Hint}: One way to compute the binomial coefficients is to use
SciPy module, which is a collection of Python-based software for math,
probability, statistics, science, and enginnering. Feel free to use
\href{https://docs.scipy.org/doc/scipy/reference/generated/scipy.special.comb.html\#scipy.special.comb}{scipy.special.comb}**

    \begin{Verbatim}[commandchars=\\\{\}]
{\color{incolor}In [{\color{incolor}79}]:} \PY{k+kn}{from} \PY{n+nn}{scipy} \PY{k}{import} \PY{n}{special}
         
         \PY{k}{def} \PY{n+nf}{prob\PYZus{}at\PYZus{}most}\PY{p}{(}\PY{n}{n}\PY{p}{,} \PY{n}{p}\PY{p}{,} \PY{n}{s}\PY{p}{)}\PY{p}{:}
             \PY{l+s+sd}{\PYZdq{}\PYZdq{}\PYZdq{} }
         \PY{l+s+sd}{    returns the probability of S \PYZlt{}= s}
         \PY{l+s+sd}{    Input n: sample size; p : proportion; s: number of successes at most}
         \PY{l+s+sd}{    \PYZdq{}\PYZdq{}\PYZdq{}}
             \PY{c+c1}{\PYZsh{} BEGIN YOUR CODE}
             \PY{c+c1}{\PYZsh{} \PYZhy{}\PYZhy{}\PYZhy{}\PYZhy{}\PYZhy{}\PYZhy{}\PYZhy{}\PYZhy{}\PYZhy{}\PYZhy{}\PYZhy{}\PYZhy{}\PYZhy{}\PYZhy{}\PYZhy{}\PYZhy{}\PYZhy{}\PYZhy{}\PYZhy{}\PYZhy{}\PYZhy{}\PYZhy{}\PYZhy{}}
             \PY{k}{if}\PY{p}{(}\PY{n}{p} \PY{o}{\PYZgt{}} \PY{l+m+mi}{1} \PY{o+ow}{or} \PY{n}{s} \PY{o}{\PYZgt{}} \PY{n}{n}\PY{p}{)}\PY{p}{:}
                 \PY{k}{return} \PY{l+m+mi}{0}
             \PY{n}{total} \PY{o}{=} \PY{l+m+mi}{0}
             \PY{k}{for} \PY{n}{i} \PY{o+ow}{in} \PY{n+nb}{range}\PY{p}{(}\PY{n}{s} \PY{o}{+} \PY{l+m+mi}{1}\PY{p}{)}\PY{p}{:}
                 \PY{n}{total} \PY{o}{+}\PY{o}{=} \PY{p}{(}\PY{n}{special}\PY{o}{.}\PY{n}{comb}\PY{p}{(}\PY{n}{n}\PY{p}{,}\PY{n}{i}\PY{p}{)} \PY{o}{*} \PY{n}{p}\PY{o}{*}\PY{o}{*}\PY{n}{i} \PY{o}{*} \PY{p}{(}\PY{l+m+mi}{1}\PY{o}{\PYZhy{}}\PY{n}{p}\PY{p}{)}\PY{o}{*}\PY{o}{*}\PY{p}{(}\PY{n}{n}\PY{o}{\PYZhy{}}\PY{n}{i}\PY{p}{)}\PY{p}{)}
             \PY{k}{return} \PY{n}{total}
             \PY{c+c1}{\PYZsh{} \PYZhy{}\PYZhy{}\PYZhy{}\PYZhy{}\PYZhy{}\PYZhy{}\PYZhy{}\PYZhy{}\PYZhy{}\PYZhy{}\PYZhy{}\PYZhy{}\PYZhy{}\PYZhy{}\PYZhy{}\PYZhy{}\PYZhy{}\PYZhy{}\PYZhy{}\PYZhy{}\PYZhy{}\PYZhy{}\PYZhy{}}
             \PY{c+c1}{\PYZsh{} END YOUR CODE}
\end{Verbatim}


    \begin{Verbatim}[commandchars=\\\{\}]
{\color{incolor}In [{\color{incolor}80}]:} \PY{n}{ok}\PY{o}{.}\PY{n}{grade}\PY{p}{(}\PY{l+s+s2}{\PYZdq{}}\PY{l+s+s2}{q6a}\PY{l+s+s2}{\PYZdq{}}\PY{p}{)}\PY{p}{;}
\end{Verbatim}


    
    \begin{verbatim}
<okgrade.result.TestResult at 0x1125d7c88>
    \end{verbatim}

    
    \subsubsection{Question 6b (1pt)}\label{question-6b-1pt}

In an election, supporters of Candidate C are in a minority. Only 45\%
of the voters in the population favor the candidate.

Suppose a survey organization takes a sample of 200 voters at random
with replacement from this population. Use \texttt{prob\_at\_most} to
write an expression that evaluates to the chance that a majority (more
than half) of the sampled voters favor Candidate C.

    \begin{Verbatim}[commandchars=\\\{\}]
{\color{incolor}In [{\color{incolor}97}]:} \PY{c+c1}{\PYZsh{} BEGIN YOUR CODE}
         \PY{c+c1}{\PYZsh{} \PYZhy{}\PYZhy{}\PYZhy{}\PYZhy{}\PYZhy{}\PYZhy{}\PYZhy{}\PYZhy{}\PYZhy{}\PYZhy{}\PYZhy{}\PYZhy{}\PYZhy{}\PYZhy{}\PYZhy{}\PYZhy{}\PYZhy{}\PYZhy{}\PYZhy{}\PYZhy{}\PYZhy{}\PYZhy{}\PYZhy{}}
         \PY{n}{p\PYZus{}majority} \PY{o}{=} \PY{n}{prob\PYZus{}at\PYZus{}most}\PY{p}{(}\PY{l+m+mi}{200}\PY{p}{,}\PY{l+m+mf}{0.45}\PY{p}{,}\PY{l+m+mi}{101}\PY{p}{)}
         \PY{c+c1}{\PYZsh{} \PYZhy{}\PYZhy{}\PYZhy{}\PYZhy{}\PYZhy{}\PYZhy{}\PYZhy{}\PYZhy{}\PYZhy{}\PYZhy{}\PYZhy{}\PYZhy{}\PYZhy{}\PYZhy{}\PYZhy{}\PYZhy{}\PYZhy{}\PYZhy{}\PYZhy{}\PYZhy{}\PYZhy{}\PYZhy{}\PYZhy{}}
         \PY{c+c1}{\PYZsh{} END YOUR CODE}
\end{Verbatim}


    \begin{Verbatim}[commandchars=\\\{\}]
{\color{incolor}In [{\color{incolor}98}]:} \PY{n}{ok}\PY{o}{.}\PY{n}{grade}\PY{p}{(}\PY{l+s+s2}{\PYZdq{}}\PY{l+s+s2}{q6b}\PY{l+s+s2}{\PYZdq{}}\PY{p}{)}\PY{p}{;}
\end{Verbatim}


    
    \begin{verbatim}
<okgrade.result.TestResult at 0x151f334e80>
    \end{verbatim}

    
    \subsubsection{Question 6c (1pt)}\label{question-6c-1pt}

Suppose each of five survey organizations takes a sample of voters at
random with replacement from the population of voters in Part
\textbf{b}, independently of the samples drawn by the other
organizations.

\begin{itemize}
\tightlist
\item
  Three of the organizations use a sample size of 200
\item
  One organization uses a sample size of 300
\item
  One organization uses a sample size of 400
\end{itemize}

Write an expression that evaluates to the chance that in at least one of
the five samples the majority of voters favor Candidate C. You can use
any quantity or function defined earlier in this exercise.

    \begin{Verbatim}[commandchars=\\\{\}]
{\color{incolor}In [{\color{incolor}136}]:} \PY{c+c1}{\PYZsh{} BEGIN YOUR CODE}
          \PY{c+c1}{\PYZsh{} \PYZhy{}\PYZhy{}\PYZhy{}\PYZhy{}\PYZhy{}\PYZhy{}\PYZhy{}\PYZhy{}\PYZhy{}\PYZhy{}\PYZhy{}\PYZhy{}\PYZhy{}\PYZhy{}\PYZhy{}\PYZhy{}\PYZhy{}\PYZhy{}\PYZhy{}\PYZhy{}\PYZhy{}\PYZhy{}\PYZhy{}}
          \PY{n}{prob\PYZus{}of\PYZus{}favor\PYZus{}C} \PY{o}{=} \PY{p}{[}\PY{n}{prob\PYZus{}at\PYZus{}most}\PY{p}{(}\PY{n}{i}\PY{p}{,} \PY{l+m+mf}{0.45}\PY{p}{,} \PY{n+nb}{int}\PY{p}{(}\PY{n}{i}\PY{o}{/}\PY{l+m+mi}{2}\PY{p}{)}\PY{o}{+}\PY{l+m+mi}{1}\PY{p}{)} \PY{k}{for} \PY{n}{i} \PY{o+ow}{in} \PY{p}{[}\PY{l+m+mi}{200}\PY{p}{,} \PY{l+m+mi}{200}\PY{p}{,} \PY{l+m+mi}{200}\PY{p}{,} \PY{l+m+mi}{300}\PY{p}{,} \PY{l+m+mi}{400}\PY{p}{]}\PY{p}{]}
          \PY{n}{prob\PYZus{}of\PYZus{}no\PYZus{}favor\PYZus{}C} \PY{o}{=} \PY{p}{[}\PY{l+m+mi}{1}\PY{o}{\PYZhy{}}\PY{n}{x} \PY{k}{for} \PY{n}{x} \PY{o+ow}{in} \PY{n}{prob\PYZus{}of\PYZus{}favor\PYZus{}C}\PY{p}{]}
          \PY{n}{prob\PYZus{}of\PYZus{}no\PYZus{}C} \PY{o}{=} \PY{n}{prob\PYZus{}of\PYZus{}no\PYZus{}favor\PYZus{}C}\PY{p}{[}\PY{l+m+mi}{0}\PY{p}{]}\PY{o}{*}\PY{n}{prob\PYZus{}of\PYZus{}no\PYZus{}favor\PYZus{}C}\PY{p}{[}\PY{l+m+mi}{1}\PY{p}{]}\PY{o}{*}\PY{n}{prob\PYZus{}of\PYZus{}no\PYZus{}favor\PYZus{}C}\PY{p}{[}\PY{l+m+mi}{2}\PY{p}{]}\PY{o}{*}\PY{n}{prob\PYZus{}of\PYZus{}no\PYZus{}favor\PYZus{}C}\PY{p}{[}\PY{l+m+mi}{3}\PY{p}{]}\PY{o}{*}\PY{n}{prob\PYZus{}of\PYZus{}no\PYZus{}favor\PYZus{}C}\PY{p}{[}\PY{l+m+mi}{4}\PY{p}{]}
          \PY{n}{prob\PYZus{}6c} \PY{o}{=} \PY{l+m+mi}{1} \PY{o}{\PYZhy{}} \PY{n}{prob\PYZus{}of\PYZus{}no\PYZus{}C}
          \PY{c+c1}{\PYZsh{} \PYZhy{}\PYZhy{}\PYZhy{}\PYZhy{}\PYZhy{}\PYZhy{}\PYZhy{}\PYZhy{}\PYZhy{}\PYZhy{}\PYZhy{}\PYZhy{}\PYZhy{}\PYZhy{}\PYZhy{}\PYZhy{}\PYZhy{}\PYZhy{}\PYZhy{}\PYZhy{}\PYZhy{}\PYZhy{}\PYZhy{}}
          \PY{c+c1}{\PYZsh{} END YOUR CODE}
\end{Verbatim}


    \begin{Verbatim}[commandchars=\\\{\}]
{\color{incolor}In [{\color{incolor}137}]:} \PY{n}{ok}\PY{o}{.}\PY{n}{grade}\PY{p}{(}\PY{l+s+s2}{\PYZdq{}}\PY{l+s+s2}{q6c}\PY{l+s+s2}{\PYZdq{}}\PY{p}{)}\PY{p}{;}
\end{Verbatim}


    \begin{Verbatim}[commandchars=\\\{\}]

        ---------------------------------------------------------------------------

        AssertionError                            Traceback (most recent call last)

        <ipython-input-137-45e1533f8d16> in <module>()
    ----> 1 ok.grade("q6c");
    

        \textasciitilde{}/anaconda3/envs/data100/lib/python3.6/site-packages/client/api/notebook.py in grade(self, question, global\_env)
         56             \# inspect trick to pass in its parents' global env.
         57             global\_env = inspect.currentframe().f\_back.f\_globals
    ---> 58         result = grade(path, global\_env)
         59         \# We display the output if we're in IPython.
         60         \# This keeps backwards compatibility with okpy's grade method


        \textasciitilde{}/anaconda3/envs/data100/lib/python3.6/site-packages/okgrade/grader.py in grade(test\_file\_path, global\_env)
         59     Returns a TestResult object.
         60     """
    ---> 61     tests = parse\_ok\_test(test\_file\_path)
         62     if global\_env is None:
         63         \# Get the global env of our callers - one level below us in the stack


        \textasciitilde{}/anaconda3/envs/data100/lib/python3.6/site-packages/okgrade/grader.py in parse\_ok\_test(path)
         24 
         25     \# Do not support point values other than 1
    ---> 26     assert test\_spec.get('points', 1) == 1
         27 
         28     test\_suite = test\_spec['suites'][0]


        AssertionError: 

    \end{Verbatim}

    \subsection{The US Presidential
Election}\label{the-us-presidential-election}

The US president is chosen by the Electoral College, not by the popular
vote. Each state is alotted a certain number of electoral college votes,
as a function of their population. Whomever wins in the state gets all
of the electoral college votes for that state.

There are 538 electoral college votes (hence the name of the Nate
Silver's site, FiveThirtyEight).

Pollsters correctly predicted the election outcome in 46 of the 50
states. For these 46 states Trump received 231 and Clinton received 232
electoral college votes.

The remaining 4 states accounted for a total of 75 votes, and whichever
candidate received the majority of the electoral college votes in these
states would win the election.

These states were Florida, Michigan, Pennsylvania, and Wisconsin.

\begin{longtable}[]{@{}ll@{}}
\toprule
State & Electoral College Votes\tabularnewline
\midrule
\endhead
Florida & 29\tabularnewline
Michigan & 16\tabularnewline
Pennsylvania & 20\tabularnewline
Wisconsin & 10\tabularnewline
\bottomrule
\end{longtable}

For Donald Trump to win the election, he had to win either: * Florida +
one (or more) other states * Michigan, Pennsylvania, and Wisconsin

The electoral margins were very narrow in these four states, as seen
below:

\begin{longtable}[]{@{}llll@{}}
\toprule
State & Trump & Clinton & Total Voters\tabularnewline
\midrule
\endhead
Florida & 49.02 & 47.82 & 9,419,886\tabularnewline
Michigan & 47.50 & 47.27 & 4,799,284\tabularnewline
Pennsylvania & 48.18 & 47.46 & 6,165,478\tabularnewline
Wisconsin & 47.22 & 46.45 & 2,976,150\tabularnewline
\bottomrule
\end{longtable}

Those narrow electoral margins can make it hard to predict the outcome
given the sample sizes that the polls used.

    \subsubsection{Question 7a (1pt)}\label{question-7a-1pt}

For your convenience, the results of the vote in the four pivotal states
is repeated below:

\begin{longtable}[]{@{}llll@{}}
\toprule
State & Trump & Clinton & Total Voters\tabularnewline
\midrule
\endhead
Florida & 49.02 & 47.82 & 9,419,886\tabularnewline
Michigan & 47.50 & 47.27 & 4,799,284\tabularnewline
Pennsylvania & 48.18 & 47.46 & 6,165,478\tabularnewline
Wisconsin & 47.22 & 46.45 & 2,976,150\tabularnewline
\bottomrule
\end{longtable}

Using the table above, write a function
\texttt{draw\_state\_sample(N,\ state)} that returns a sample with
replacement of N voters from the given state. Your result should be
returned as a list, where the first element is the number of Trump
votes, the second element is the number of Clinton votes, and the third
is the number of Other votes. For example,
\texttt{draw\_state\_sample(1500,\ "florida")} could return
\texttt{{[}727,\ 692,\ 81{]}}. You may assume that the state name is
given in all lower case.

You might find \texttt{np.random.multinomial} useful.

    \begin{Verbatim}[commandchars=\\\{\}]
{\color{incolor}In [{\color{incolor}189}]:} \PY{k}{def} \PY{n+nf}{draw\PYZus{}state\PYZus{}sample}\PY{p}{(}\PY{n}{N}\PY{p}{,} \PY{n}{state}\PY{p}{)}\PY{p}{:}
              \PY{c+c1}{\PYZsh{} BEGIN YOUR CODE}
              \PY{c+c1}{\PYZsh{} \PYZhy{}\PYZhy{}\PYZhy{}\PYZhy{}\PYZhy{}\PYZhy{}\PYZhy{}\PYZhy{}\PYZhy{}\PYZhy{}\PYZhy{}\PYZhy{}\PYZhy{}\PYZhy{}\PYZhy{}\PYZhy{}\PYZhy{}\PYZhy{}\PYZhy{}\PYZhy{}\PYZhy{}\PYZhy{}\PYZhy{}}
              \PY{k}{if} \PY{n}{state} \PY{o}{==} \PY{l+s+s1}{\PYZsq{}}\PY{l+s+s1}{florida}\PY{l+s+s1}{\PYZsq{}}\PY{p}{:}
                  \PY{n}{a}\PY{p}{,} \PY{n}{b} \PY{o}{=} \PY{l+m+mf}{49.02}\PY{o}{/}\PY{l+m+mi}{100}\PY{p}{,} \PY{l+m+mf}{47.82}\PY{o}{/}\PY{l+m+mi}{100}
              \PY{k}{elif} \PY{n}{state} \PY{o}{==} \PY{l+s+s1}{\PYZsq{}}\PY{l+s+s1}{michigan}\PY{l+s+s1}{\PYZsq{}}\PY{p}{:}
                  \PY{n}{a}\PY{p}{,} \PY{n}{b} \PY{o}{=} \PY{l+m+mf}{47.50}\PY{o}{/}\PY{l+m+mi}{100}\PY{p}{,} \PY{l+m+mf}{47.27}\PY{o}{/}\PY{l+m+mi}{100}
              \PY{k}{elif} \PY{n}{state} \PY{o}{==} \PY{l+s+s1}{\PYZsq{}}\PY{l+s+s1}{pennsylvania}\PY{l+s+s1}{\PYZsq{}}\PY{p}{:}
                  \PY{n}{a}\PY{p}{,} \PY{n}{b} \PY{o}{=} \PY{l+m+mf}{48.18}\PY{o}{/}\PY{l+m+mi}{100}\PY{p}{,} \PY{l+m+mf}{47.46}\PY{o}{/}\PY{l+m+mi}{100}
              \PY{k}{elif} \PY{n}{state} \PY{o}{==} \PY{l+s+s1}{\PYZsq{}}\PY{l+s+s1}{wisconsin}\PY{l+s+s1}{\PYZsq{}}\PY{p}{:}
                  \PY{n}{a}\PY{p}{,} \PY{n}{b} \PY{o}{=} \PY{l+m+mf}{47.22}\PY{o}{/}\PY{l+m+mi}{100}\PY{p}{,} \PY{l+m+mf}{46.45}\PY{o}{/}\PY{l+m+mi}{100}
              \PY{n}{c} \PY{o}{=} \PY{l+m+mi}{1} \PY{o}{\PYZhy{}} \PY{p}{(}\PY{n}{a} \PY{o}{+} \PY{n}{b}\PY{p}{)}
              \PY{k}{return} \PY{n}{np}\PY{o}{.}\PY{n}{random}\PY{o}{.}\PY{n}{multinomial}\PY{p}{(}\PY{n}{N}\PY{p}{,} \PY{p}{[}\PY{n}{a}\PY{p}{,}\PY{n}{b}\PY{p}{,}\PY{n}{c}\PY{p}{]}\PY{p}{)}
              \PY{c+c1}{\PYZsh{} \PYZhy{}\PYZhy{}\PYZhy{}\PYZhy{}\PYZhy{}\PYZhy{}\PYZhy{}\PYZhy{}\PYZhy{}\PYZhy{}\PYZhy{}\PYZhy{}\PYZhy{}\PYZhy{}\PYZhy{}\PYZhy{}\PYZhy{}\PYZhy{}\PYZhy{}\PYZhy{}\PYZhy{}\PYZhy{}\PYZhy{}}
              \PY{c+c1}{\PYZsh{} END YOUR CODE}
\end{Verbatim}


    \begin{Verbatim}[commandchars=\\\{\}]
{\color{incolor}In [{\color{incolor}190}]:} \PY{n}{ok}\PY{o}{.}\PY{n}{grade}\PY{p}{(}\PY{l+s+s2}{\PYZdq{}}\PY{l+s+s2}{q7a}\PY{l+s+s2}{\PYZdq{}}\PY{p}{)}\PY{p}{;}
\end{Verbatim}


    \begin{Verbatim}[commandchars=\\\{\}]

        ---------------------------------------------------------------------------

        AssertionError                            Traceback (most recent call last)

        <ipython-input-190-8f8211f1239a> in <module>()
    ----> 1 ok.grade("q7a");
    

        \textasciitilde{}/anaconda3/envs/data100/lib/python3.6/site-packages/client/api/notebook.py in grade(self, question, global\_env)
         56             \# inspect trick to pass in its parents' global env.
         57             global\_env = inspect.currentframe().f\_back.f\_globals
    ---> 58         result = grade(path, global\_env)
         59         \# We display the output if we're in IPython.
         60         \# This keeps backwards compatibility with okpy's grade method


        \textasciitilde{}/anaconda3/envs/data100/lib/python3.6/site-packages/okgrade/grader.py in grade(test\_file\_path, global\_env)
         59     Returns a TestResult object.
         60     """
    ---> 61     tests = parse\_ok\_test(test\_file\_path)
         62     if global\_env is None:
         63         \# Get the global env of our callers - one level below us in the stack


        \textasciitilde{}/anaconda3/envs/data100/lib/python3.6/site-packages/okgrade/grader.py in parse\_ok\_test(path)
         24 
         25     \# Do not support point values other than 1
    ---> 26     assert test\_spec.get('points', 1) == 1
         27 
         28     test\_suite = test\_spec['suites'][0]


        AssertionError: 

    \end{Verbatim}

    \subsubsection{Question 7b (1pt)}\label{question-7b-1pt}

Now, create a function \texttt{trump\_advantage} that takes in a sample
of votes (like the one returned by \texttt{draw\_state\_sample}) and
returns the difference in the proportion of votes between Trump and
Clinton. For example \texttt{trump\_advantage({[}100,\ 60,\ 40{]})}
would return \texttt{0.2}, since Trump had 50\% of the votes in this
sample and Clinton had 30\%.

    \begin{Verbatim}[commandchars=\\\{\}]
{\color{incolor}In [{\color{incolor}193}]:} \PY{k}{def} \PY{n+nf}{trump\PYZus{}advantage}\PY{p}{(}\PY{n}{voter\PYZus{}sample}\PY{p}{)}\PY{p}{:}
              \PY{c+c1}{\PYZsh{} BEGIN YOUR CODE}
              \PY{c+c1}{\PYZsh{} \PYZhy{}\PYZhy{}\PYZhy{}\PYZhy{}\PYZhy{}\PYZhy{}\PYZhy{}\PYZhy{}\PYZhy{}\PYZhy{}\PYZhy{}\PYZhy{}\PYZhy{}\PYZhy{}\PYZhy{}\PYZhy{}\PYZhy{}\PYZhy{}\PYZhy{}\PYZhy{}\PYZhy{}\PYZhy{}\PYZhy{}}
              \PY{k}{return} \PY{p}{(}\PY{n}{voter\PYZus{}sample}\PY{p}{[}\PY{l+m+mi}{0}\PY{p}{]} \PY{o}{\PYZhy{}} \PY{n}{voter\PYZus{}sample}\PY{p}{[}\PY{l+m+mi}{1}\PY{p}{]}\PY{p}{)} \PY{o}{/} \PY{n+nb}{sum}\PY{p}{(}\PY{n}{voter\PYZus{}sample}\PY{p}{)}
              \PY{c+c1}{\PYZsh{} \PYZhy{}\PYZhy{}\PYZhy{}\PYZhy{}\PYZhy{}\PYZhy{}\PYZhy{}\PYZhy{}\PYZhy{}\PYZhy{}\PYZhy{}\PYZhy{}\PYZhy{}\PYZhy{}\PYZhy{}\PYZhy{}\PYZhy{}\PYZhy{}\PYZhy{}\PYZhy{}\PYZhy{}\PYZhy{}\PYZhy{}}
              \PY{c+c1}{\PYZsh{} END YOUR CODE}
\end{Verbatim}


    \begin{Verbatim}[commandchars=\\\{\}]
{\color{incolor}In [{\color{incolor}194}]:} \PY{n}{ok}\PY{o}{.}\PY{n}{grade}\PY{p}{(}\PY{l+s+s2}{\PYZdq{}}\PY{l+s+s2}{q7b}\PY{l+s+s2}{\PYZdq{}}\PY{p}{)}\PY{p}{;}
\end{Verbatim}


    
    \begin{verbatim}
<okgrade.result.TestResult at 0x151f39d198>
    \end{verbatim}

    
    \subsubsection{Question 7c (1pt)}\label{question-7c-1pt}

Simulate Trump's advantage across 100,000 samples of 1500 voters for the
state of Pennsylvania and store the results of each simulation in a list
called \texttt{simulations}.

That is, \texttt{simulations{[}i{]}} should be Trump's proportion
advantage for the \texttt{i+1}th simple random sample.

    \begin{Verbatim}[commandchars=\\\{\}]
{\color{incolor}In [{\color{incolor}197}]:} \PY{c+c1}{\PYZsh{} BEGIN YOUR CODE}
          \PY{c+c1}{\PYZsh{} \PYZhy{}\PYZhy{}\PYZhy{}\PYZhy{}\PYZhy{}\PYZhy{}\PYZhy{}\PYZhy{}\PYZhy{}\PYZhy{}\PYZhy{}\PYZhy{}\PYZhy{}\PYZhy{}\PYZhy{}\PYZhy{}\PYZhy{}\PYZhy{}\PYZhy{}\PYZhy{}\PYZhy{}\PYZhy{}\PYZhy{}}
          \PY{n}{simulations} \PY{o}{=} \PY{p}{[}\PY{n}{trump\PYZus{}advantage}\PY{p}{(}\PY{n}{draw\PYZus{}state\PYZus{}sample}\PY{p}{(}\PY{l+m+mi}{1500}\PY{p}{,} \PY{l+s+s1}{\PYZsq{}}\PY{l+s+s1}{pennsylvania}\PY{l+s+s1}{\PYZsq{}}\PY{p}{)}\PY{p}{)} \PY{k}{for} \PY{n}{i} \PY{o+ow}{in} \PY{n+nb}{range}\PY{p}{(}\PY{l+m+mi}{100000}\PY{p}{)}\PY{p}{]}
          \PY{c+c1}{\PYZsh{} \PYZhy{}\PYZhy{}\PYZhy{}\PYZhy{}\PYZhy{}\PYZhy{}\PYZhy{}\PYZhy{}\PYZhy{}\PYZhy{}\PYZhy{}\PYZhy{}\PYZhy{}\PYZhy{}\PYZhy{}\PYZhy{}\PYZhy{}\PYZhy{}\PYZhy{}\PYZhy{}\PYZhy{}\PYZhy{}\PYZhy{}}
          \PY{c+c1}{\PYZsh{} END YOUR CODE}
\end{Verbatim}


    \begin{Verbatim}[commandchars=\\\{\}]
{\color{incolor}In [{\color{incolor}198}]:} \PY{n}{ok}\PY{o}{.}\PY{n}{grade}\PY{p}{(}\PY{l+s+s2}{\PYZdq{}}\PY{l+s+s2}{q7c}\PY{l+s+s2}{\PYZdq{}}\PY{p}{)}\PY{p}{;}
\end{Verbatim}


    
    \begin{verbatim}
<okgrade.result.TestResult at 0x151f39d908>
    \end{verbatim}

    
    \subsubsection{Question 7d (1pt)}\label{question-7d-1pt}

Now write a function \texttt{trump\_wins(N)} that creates a sample of N
voters for each of the four crucial states (Florida, Michigan,
Pennsylvania, and Wisconsin) and returns 1 if Trump is predicted to win
based on these samples and 0 if Trump is predicted to lose.

Recall that for Trump to win the election, he must either: * Win the
state of Florida and 1 or more other states * Win Michigan,
Pennsylvania, and Wisconsin

    \begin{Verbatim}[commandchars=\\\{\}]
{\color{incolor}In [{\color{incolor}219}]:} \PY{k}{def} \PY{n+nf}{trump\PYZus{}wins}\PY{p}{(}\PY{n}{N}\PY{p}{)}\PY{p}{:}
              \PY{c+c1}{\PYZsh{} BEGIN YOUR CODE}
              \PY{c+c1}{\PYZsh{} \PYZhy{}\PYZhy{}\PYZhy{}\PYZhy{}\PYZhy{}\PYZhy{}\PYZhy{}\PYZhy{}\PYZhy{}\PYZhy{}\PYZhy{}\PYZhy{}\PYZhy{}\PYZhy{}\PYZhy{}\PYZhy{}\PYZhy{}\PYZhy{}\PYZhy{}\PYZhy{}\PYZhy{}\PYZhy{}\PYZhy{}}
              \PY{n}{f} \PY{o}{=} \PY{n}{trump\PYZus{}advantage}\PY{p}{(}\PY{n}{draw\PYZus{}state\PYZus{}sample}\PY{p}{(}\PY{n}{N}\PY{p}{,} \PY{l+s+s1}{\PYZsq{}}\PY{l+s+s1}{florida}\PY{l+s+s1}{\PYZsq{}}\PY{p}{)}\PY{p}{)}
              \PY{n}{m} \PY{o}{=} \PY{n}{trump\PYZus{}advantage}\PY{p}{(}\PY{n}{draw\PYZus{}state\PYZus{}sample}\PY{p}{(}\PY{n}{N}\PY{p}{,} \PY{l+s+s1}{\PYZsq{}}\PY{l+s+s1}{michigan}\PY{l+s+s1}{\PYZsq{}}\PY{p}{)}\PY{p}{)}
              \PY{n}{p} \PY{o}{=} \PY{n}{trump\PYZus{}advantage}\PY{p}{(}\PY{n}{draw\PYZus{}state\PYZus{}sample}\PY{p}{(}\PY{n}{N}\PY{p}{,} \PY{l+s+s1}{\PYZsq{}}\PY{l+s+s1}{pennsylvania}\PY{l+s+s1}{\PYZsq{}}\PY{p}{)}\PY{p}{)}
              \PY{n}{w} \PY{o}{=} \PY{n}{trump\PYZus{}advantage}\PY{p}{(}\PY{n}{draw\PYZus{}state\PYZus{}sample}\PY{p}{(}\PY{n}{N}\PY{p}{,} \PY{l+s+s1}{\PYZsq{}}\PY{l+s+s1}{wisconsin}\PY{l+s+s1}{\PYZsq{}}\PY{p}{)}\PY{p}{)}
              \PY{k}{if} \PY{p}{(}\PY{n}{f}\PY{o}{\PYZgt{}}\PY{l+m+mi}{0} \PY{o+ow}{and} \PY{p}{(}\PY{n}{m}\PY{o}{\PYZgt{}}\PY{l+m+mi}{0} \PY{o+ow}{or} \PY{n}{p}\PY{o}{\PYZgt{}}\PY{l+m+mi}{0} \PY{o+ow}{or} \PY{n}{w}\PY{o}{\PYZgt{}}\PY{l+m+mi}{0}\PY{p}{)}\PY{p}{)} \PY{o+ow}{or} \PY{p}{(}\PY{n}{m}\PY{o}{\PYZgt{}}\PY{l+m+mi}{0} \PY{o+ow}{and} \PY{n}{p}\PY{o}{\PYZgt{}}\PY{l+m+mi}{0} \PY{o+ow}{and} \PY{n}{w}\PY{o}{\PYZgt{}}\PY{l+m+mi}{0}\PY{p}{)}\PY{p}{:}
                  \PY{k}{return} \PY{l+m+mi}{1}
              \PY{k}{return} \PY{l+m+mi}{0}
              \PY{c+c1}{\PYZsh{} \PYZhy{}\PYZhy{}\PYZhy{}\PYZhy{}\PYZhy{}\PYZhy{}\PYZhy{}\PYZhy{}\PYZhy{}\PYZhy{}\PYZhy{}\PYZhy{}\PYZhy{}\PYZhy{}\PYZhy{}\PYZhy{}\PYZhy{}\PYZhy{}\PYZhy{}\PYZhy{}\PYZhy{}\PYZhy{}\PYZhy{}}
              \PY{c+c1}{\PYZsh{} END YOUR CODE}
\end{Verbatim}


    \begin{Verbatim}[commandchars=\\\{\}]
{\color{incolor}In [{\color{incolor}220}]:} \PY{n}{ok}\PY{o}{.}\PY{n}{grade}\PY{p}{(}\PY{l+s+s2}{\PYZdq{}}\PY{l+s+s2}{q7d}\PY{l+s+s2}{\PYZdq{}}\PY{p}{)}\PY{p}{;}
\end{Verbatim}


    \begin{Verbatim}[commandchars=\\\{\}]

        ---------------------------------------------------------------------------

        AssertionError                            Traceback (most recent call last)

        <ipython-input-220-8daf74a6ad4d> in <module>()
    ----> 1 ok.grade("q7d");
    

        \textasciitilde{}/anaconda3/envs/data100/lib/python3.6/site-packages/client/api/notebook.py in grade(self, question, global\_env)
         56             \# inspect trick to pass in its parents' global env.
         57             global\_env = inspect.currentframe().f\_back.f\_globals
    ---> 58         result = grade(path, global\_env)
         59         \# We display the output if we're in IPython.
         60         \# This keeps backwards compatibility with okpy's grade method


        \textasciitilde{}/anaconda3/envs/data100/lib/python3.6/site-packages/okgrade/grader.py in grade(test\_file\_path, global\_env)
         59     Returns a TestResult object.
         60     """
    ---> 61     tests = parse\_ok\_test(test\_file\_path)
         62     if global\_env is None:
         63         \# Get the global env of our callers - one level below us in the stack


        \textasciitilde{}/anaconda3/envs/data100/lib/python3.6/site-packages/okgrade/grader.py in parse\_ok\_test(path)
         24 
         25     \# Do not support point values other than 1
    ---> 26     assert test\_spec.get('points', 1) == 1
         27 
         28     test\_suite = test\_spec['suites'][0]


        AssertionError: 

    \end{Verbatim}

    \subsubsection{Question 7e}\label{question-7e}

If we repeat 100,000 simulations of the election, i.e. we call
\texttt{trump\_wins(1500)} 100,000 times, what proportion of these
simulations predict a Trump victory? Give your answer as
\texttt{proportion\_trump}.

This number represents the percent chance that a given sample will
correctly predict Trump's victory \emph{even if the sample was collected
with absolutely no bias}.

    \begin{Verbatim}[commandchars=\\\{\}]
{\color{incolor}In [{\color{incolor}225}]:} \PY{c+c1}{\PYZsh{} BEGIN YOUR CODE}
          \PY{c+c1}{\PYZsh{} \PYZhy{}\PYZhy{}\PYZhy{}\PYZhy{}\PYZhy{}\PYZhy{}\PYZhy{}\PYZhy{}\PYZhy{}\PYZhy{}\PYZhy{}\PYZhy{}\PYZhy{}\PYZhy{}\PYZhy{}\PYZhy{}\PYZhy{}\PYZhy{}\PYZhy{}\PYZhy{}\PYZhy{}\PYZhy{}\PYZhy{}}
          \PY{n}{proportion\PYZus{}trump} \PY{o}{=} \PY{n+nb}{sum}\PY{p}{(}\PY{p}{[}\PY{n}{trump\PYZus{}wins}\PY{p}{(}\PY{l+m+mi}{1500}\PY{p}{)} \PY{k}{for} \PY{n}{i} \PY{o+ow}{in} \PY{n+nb}{range}\PY{p}{(}\PY{l+m+mi}{100000}\PY{p}{)}\PY{p}{]}\PY{p}{)}\PY{o}{/}\PY{l+m+mi}{100000}
          \PY{c+c1}{\PYZsh{} \PYZhy{}\PYZhy{}\PYZhy{}\PYZhy{}\PYZhy{}\PYZhy{}\PYZhy{}\PYZhy{}\PYZhy{}\PYZhy{}\PYZhy{}\PYZhy{}\PYZhy{}\PYZhy{}\PYZhy{}\PYZhy{}\PYZhy{}\PYZhy{}\PYZhy{}\PYZhy{}\PYZhy{}\PYZhy{}\PYZhy{}}
          \PY{c+c1}{\PYZsh{} END YOUR CODE}
          \PY{n}{proportion\PYZus{}trump}
\end{Verbatim}


\begin{Verbatim}[commandchars=\\\{\}]
{\color{outcolor}Out[{\color{outcolor}225}]:} 0.69423
\end{Verbatim}
            
    \begin{Verbatim}[commandchars=\\\{\}]
{\color{incolor}In [{\color{incolor}226}]:} \PY{n}{ok}\PY{o}{.}\PY{n}{grade}\PY{p}{(}\PY{l+s+s2}{\PYZdq{}}\PY{l+s+s2}{q7e}\PY{l+s+s2}{\PYZdq{}}\PY{p}{)}\PY{p}{;}
\end{Verbatim}


    
    \begin{verbatim}
<okgrade.result.TestResult at 0x1125935f8>
    \end{verbatim}

    
    \subsubsection{Congratulations! You have completed
HW1.}\label{congratulations-you-have-completed-hw1.}

Make sure you have run all cells in your notebook in order before
running the cell below, so that all images/graphs appear in the output.,

\textbf{Please save before submitting!}

    For your convenience, you can run this cell to run all the tests at
once!

    \begin{Verbatim}[commandchars=\\\{\}]
{\color{incolor}In [{\color{incolor}227}]:} \PY{k+kn}{import} \PY{n+nn}{os}
          \PY{k+kn}{from} \PY{n+nn}{IPython}\PY{n+nn}{.}\PY{n+nn}{utils} \PY{k}{import} \PY{n}{io}
          
          \PY{n+nb}{print}\PY{p}{(}\PY{l+s+s1}{\PYZsq{}}\PY{l+s+si}{\PYZob{}:5\PYZcb{}}\PY{l+s+s1}{|}\PY{l+s+se}{\PYZbs{}t}\PY{l+s+si}{\PYZob{}:6\PYZcb{}}\PY{l+s+s1}{|}\PY{l+s+se}{\PYZbs{}t}\PY{l+s+si}{\PYZob{}:6\PYZcb{}}\PY{l+s+s1}{\PYZsq{}}\PY{o}{.}\PY{n}{format}\PY{p}{(}\PY{l+s+s1}{\PYZsq{}}\PY{l+s+s1}{Q}\PY{l+s+s1}{\PYZsq{}}\PY{p}{,} \PY{l+s+s1}{\PYZsq{}}\PY{l+s+s1}{Passed}\PY{l+s+s1}{\PYZsq{}}\PY{p}{,} \PY{l+s+s1}{\PYZsq{}}\PY{l+s+s1}{Failed}\PY{l+s+s1}{\PYZsq{}}\PY{p}{)}\PY{p}{)}
          \PY{n+nb}{print}\PY{p}{(}\PY{l+s+s1}{\PYZsq{}}\PY{l+s+s1}{\PYZhy{}\PYZhy{}\PYZhy{}\PYZhy{}\PYZhy{}\PYZhy{}\PYZhy{}\PYZhy{}\PYZhy{}\PYZhy{}\PYZhy{}\PYZhy{}\PYZhy{}\PYZhy{}\PYZhy{}\PYZhy{}\PYZhy{}\PYZhy{}\PYZhy{}\PYZhy{}\PYZhy{}\PYZhy{}\PYZhy{}\PYZhy{}}\PY{l+s+s1}{\PYZsq{}}\PY{p}{)}
          \PY{k}{for} \PY{n}{q} \PY{o+ow}{in} \PY{n+nb}{sorted}\PY{p}{(}\PY{n}{os}\PY{o}{.}\PY{n}{listdir}\PY{p}{(}\PY{l+s+s2}{\PYZdq{}}\PY{l+s+s2}{tests}\PY{l+s+s2}{\PYZdq{}}\PY{p}{)}\PY{p}{)}\PY{p}{:}
              \PY{k}{if} \PY{n}{q}\PY{o}{.}\PY{n}{startswith}\PY{p}{(}\PY{l+s+s1}{\PYZsq{}}\PY{l+s+s1}{q}\PY{l+s+s1}{\PYZsq{}}\PY{p}{)} \PY{o+ow}{and} \PY{n+nb}{len}\PY{p}{(}\PY{n}{q}\PY{p}{)} \PY{o}{\PYZlt{}}\PY{o}{=} \PY{l+m+mi}{10}\PY{p}{:}
                  \PY{k}{with} \PY{n}{io}\PY{o}{.}\PY{n}{capture\PYZus{}output}\PY{p}{(}\PY{p}{)} \PY{k}{as} \PY{n}{captured}\PY{p}{:}
                      \PY{n}{score} \PY{o}{=} \PY{n}{ok}\PY{o}{.}\PY{n}{grade}\PY{p}{(}\PY{n}{q}\PY{p}{[}\PY{p}{:}\PY{o}{\PYZhy{}}\PY{l+m+mi}{3}\PY{p}{]}\PY{p}{)}\PY{p}{;}
                  \PY{n+nb}{print}\PY{p}{(}\PY{l+s+s1}{\PYZsq{}}\PY{l+s+si}{\PYZob{}:5\PYZcb{}}\PY{l+s+s1}{|}\PY{l+s+se}{\PYZbs{}t}\PY{l+s+si}{\PYZob{}:6\PYZcb{}}\PY{l+s+s1}{|}\PY{l+s+se}{\PYZbs{}t}\PY{l+s+si}{\PYZob{}:6\PYZcb{}}\PY{l+s+s1}{\PYZsq{}}\PY{o}{.}\PY{n}{format}\PY{p}{(}\PY{n}{q}\PY{p}{[}\PY{p}{:}\PY{o}{\PYZhy{}}\PY{l+m+mi}{3}\PY{p}{]}\PY{p}{,} \PY{n}{score}\PY{p}{[}\PY{l+s+s1}{\PYZsq{}}\PY{l+s+s1}{passed}\PY{l+s+s1}{\PYZsq{}}\PY{p}{]}\PY{p}{,} \PY{n}{score}\PY{p}{[}\PY{l+s+s1}{\PYZsq{}}\PY{l+s+s1}{failed}\PY{l+s+s1}{\PYZsq{}}\PY{p}{]}\PY{p}{)}\PY{p}{)}
\end{Verbatim}


    \begin{Verbatim}[commandchars=\\\{\}]
Q    |	Passed|	Failed
------------------------

    \end{Verbatim}

    \begin{Verbatim}[commandchars=\\\{\}]

        ---------------------------------------------------------------------------

        TypeError                                 Traceback (most recent call last)

        <ipython-input-227-29bc21abedad> in <module>()
          8         with io.capture\_output() as captured:
          9             score = ok.grade(q[:-3]);
    ---> 10         print('\{:5\}|\textbackslash{}t\{:6\}|\textbackslash{}t\{:6\}'.format(q[:-3], score['passed'], score['failed']))
    

        TypeError: 'TestResult' object is not subscriptable

    \end{Verbatim}

    Please generate pdf as follows and submit it to Gradescope.

\textbf{File \textgreater{} Print Preview \textgreater{} Print
\textgreater{} Save as pdf}


    % Add a bibliography block to the postdoc
    
    
    
    \end{document}
